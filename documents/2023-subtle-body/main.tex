\documentclass{article}
\usepackage[utf8]{inputenc}
\usepackage[super,square]{natbib}
\usepackage{tabularx}
\usepackage{parskip}
\usepackage[margin=1.4in]{geometry}
\usepackage{csquotes}
\usepackage{mathrsfs}
\usepackage{amsmath}
\usepackage{amsfonts}
\usepackage{amsthm}
\usepackage{amssymb}
\usepackage{hyperref}
\usepackage{graphicx}
\usepackage{float}
\usepackage{mdframed}
\usepackage[dvipsnames]{xcolor}

\usepackage{multicol}
\usepackage{float}
% \usepackage[french]{babel} 
\usepackage{csquotes}
\usepackage[utf8]{inputenc}

% Book headers
\usepackage{fancyhdr}
\pagestyle{fancy}
\fancyhf{}
\fancyhead[L]{\rightmark}
\fancyhead[R]{\thepage}
\renewcommand{\headrulewidth}{0pt}


\definecolor{blueish}{HTML}{CAC8FA}
\definecolor{blueish}{HTML}{CAC8FA}


\newcommand{\comment}[1]{}
\newtheorem{theorem}{Theorem}[section]
\newtheorem{corollary}{Corollary}[theorem]
\newtheorem{proposition}{Proposition}[theorem]
\newtheorem{lemma}[theorem]{Lemma}
\newtheorem{identity}[theorem]{Identity}

\theoremstyle{definition}
\newtheorem{defn}[theorem]{Definition}
\newtheorem{example}[theorem]{Example}
\newenvironment{definition}
  {\vspace{8pt}\begin{mdframed}[backgroundcolor=white,linecolor=blue]\begin{defn}}
  {\end{defn}\end{mdframed}\vspace{4pt}}

\newtheorem{pce}[theorem]{Practice}
\newenvironment{practice}
  {\vspace{8pt}\begin{mdframed}[backgroundcolor=white,linecolor=green]\begin{pce}}
  {\end{pce}\end{mdframed}\vspace{4pt}}


\newtheorem{note}[theorem]{Memo}
\newenvironment{memo}
  {\vspace{8pt}\begin{mdframed}[backgroundcolor=white,linecolor=yellow]\begin{note}}
  {\end{note}\end{mdframed}\vspace{4pt}}


% \usepackage{xpatch}
% \xapptocmd{\appendix}{%
%   \counterwithin{theorem}{section}
%   \counterwithin{definition}{section}
% }{\typeout{Success}}{}

\title{\vspace{-3cm} Geometrization of the Subtle Body}
\author{Luke J Pereira}
\date{}


\begin{document}

\maketitle
\begin{abstract}
\noindent
It is proposed that the \textit{grounding} of concepts into an embodied intelligence is initially developed from synesthetic multi-modal sense samples of perception.
Moreover, it is argued that collections of synesthetic sense perception forms the basis or skeleton of what is referred to as the ``subtle body'' in the Buddhist Pali canon.
The subtle (or energy) body is a non-physical object of meditation derived from awareness of subtle sense and musculature observations that is refined and manipulated during various meditation and yoga practices, often producing altered states and generating insights.
The refinement of awareness and control of the subtle body is fundamental to the development of wisdom and moral phenomenology in Buddhist ethics, which contrasts Aristotelian virtue ethics, Kantian rule-based ethics (deontology), and British utilitarian consequentialism.
A formal description of the subtle body allows us to explore its application to the ethics and alignment of artificial intelligent agents.
A synesthetic model of the subtle body enables its geometrization through representation theory, which can naturally be applied to pre-training inductive biases in representation learning models.
The synesthetic content of interest is the shared activations among different sensory stimuli common to an experience. 
Since the encoding of sense modalities varies due to differences in biological and neural pathways, it can more generally be described as the invariants of morphisms between initial and terminal activations of different sense modalities.
% In particular, certain observed objects stimulate multiple sense doors and trigger a simultaneous experience of sight, scent, taste, touch, or sound.
Through synesthetic sensory interactions with various objects and environments, agents develop a collection of invariant representations that span all sense modalities and phenomenal range. 
This forms a category that can be de-contextualized from particular objects into its own entity of perception with a phenomenology comparable to the subtle body. 
We can formalize this object through category theory as a Skeleton category that combinds and reduces invariants from pointed categories of various initial and terminal activations.
More specifically using representation theory, we describe the object as a quiver gauge theory and explore its implementation in a geometric deep learning framework with steerable graph neural networks, mechanistic interpretability experiments, and predictive inference with conformal prediction.
% The disruption and transformation of a spanning body schema is also comparable to the disruption of GCPRs (mediating various sense perceptions) by psychedelics, which also appears to disrupt predictive coding hierarchies leading to altered states.
\end{abstract}

\end{document}
