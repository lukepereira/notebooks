\documentclass{article}
\usepackage[utf8]{inputenc}
\usepackage[super,square]{natbib}
\usepackage{tabularx}
\usepackage{parskip}
\usepackage[margin=1.4in]{geometry}
\usepackage{csquotes}
\usepackage{mathrsfs}
\usepackage{amsmath}
\usepackage{amsfonts}
\usepackage{amsthm}
\usepackage{amssymb}
\usepackage{hyperref}
\usepackage{graphicx}
\usepackage{float}
\usepackage{mdframed}
\usepackage[dvipsnames]{xcolor}

\usepackage{multicol}
\usepackage{float}
% \usepackage[french]{babel} 
\usepackage{csquotes}
\usepackage[utf8]{inputenc}

% Book headers
\usepackage{fancyhdr}
\pagestyle{fancy}
\fancyhf{}
\fancyhead[L]{\rightmark}
\fancyhead[R]{\thepage}
\renewcommand{\headrulewidth}{0pt}


\definecolor{blueish}{HTML}{CAC8FA}
\definecolor{blueish}{HTML}{CAC8FA}


\newcommand{\comment}[1]{}
\newtheorem{theorem}{Theorem}[section]
\newtheorem{corollary}{Corollary}[theorem]
\newtheorem{proposition}{Proposition}[theorem]
\newtheorem{lemma}[theorem]{Lemma}
\newtheorem{identity}[theorem]{Identity}

\theoremstyle{definition}
\newtheorem{defn}[theorem]{Definition}
\newtheorem{example}[theorem]{Example}
\newenvironment{definition}
  {\vspace{8pt}\begin{mdframed}[backgroundcolor=white,linecolor=blue]\begin{defn}}
  {\end{defn}\end{mdframed}\vspace{4pt}}

\newtheorem{pce}[theorem]{Practice}
\newenvironment{practice}
  {\vspace{8pt}\begin{mdframed}[backgroundcolor=white,linecolor=green]\begin{pce}}
  {\end{pce}\end{mdframed}\vspace{4pt}}


\newtheorem{note}[theorem]{Memo}
\newenvironment{memo}
  {\vspace{8pt}\begin{mdframed}[backgroundcolor=white,linecolor=yellow]\begin{note}}
  {\end{note}\end{mdframed}\vspace{4pt}}


% \usepackage{xpatch}
% \xapptocmd{\appendix}{%
%   \counterwithin{theorem}{section}
%   \counterwithin{definition}{section}
% }{\typeout{Success}}{}

\title{\vspace{-3cm} Geometrization of the Subtle Body}
\author{Luke J Pereira}
\date{}


\begin{document}

\maketitle
\begin{abstract}
\noindent
It is proposed that the \textit{grounding} of concepts into an embodied intelligence is developed from synesthetic multi-modal sense samples of our perception.
Moreover, it is argued that our synesthetic sense perceptions forms the basis of what is referred to as the ``subtle body'' described in Buddhist writings.
The refinement of the subtle body is fundamental to Buddhist ethics and moral phenomenology, which is in contrast Aristotelian virtue ethics, Kantian deontology or rule-based ethics, as well as British utilitarian consequentialism. 
A formalization of the subtle body allows us to explore its application to ethics alignment of artificial intelligent agents.
A synesthetic model of the subtle body allows for its geometrization through representation theory, which naturally integrates into pre-training inductive biases into representation learning models.
The synesthetic experience of interest is the shared activations related to different sensory stimuli. 
Since the encoding of senses varies from differences in biological pathways and data modalities, synesthesia can be generally described as invariants of morphisms between activated sense data of different sensory stimuli.
In particular, certain objects stimulate multiple sense doors and trigger a simultaneous experience of smell, taste, or touch.
Through synesthetic sensory interactions with various objects and environments, agents develop a collection of invariant representations that spans all sense modalities and phenomenal range. 
This forms a category that can be de-contextualized from the particular objects originally observed into its own entity of perception.
We can formalize this object through category theory as a Skeleton category and more specifically through representation theory as a quiver gauge theory.
This category of synesthetic invariants has phenomenology comparable to the subtle body. 
The subtle (or energy) body is a non-physical object of meditation that is manipulated during various practices, often producing altered states and generating insights.
The disruption and transformation of a spanning body schema is also comparable to the disruption of GCPRs (mediating various sense perceptions) by psychedelics, which also appears to disrupt predictive coding hierarchies leading to altered states.
\end{abstract}

\newpage

\end{document}
