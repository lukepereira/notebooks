\documentclass{article}
\usepackage[utf8]{inputenc}
\usepackage[super,square]{natbib}
\usepackage{tabularx}
\usepackage{parskip}
\usepackage[margin=1.4in]{geometry}
\usepackage{csquotes}
\usepackage{mathrsfs}
\usepackage{amsmath}
\usepackage{amsfonts}
\usepackage{amsthm}
\usepackage{amssymb}
\usepackage{hyperref}
\usepackage{graphicx}
\usepackage{float}
\usepackage{mdframed}
\usepackage[dvipsnames]{xcolor}

\usepackage{multicol}
\usepackage{float}
% \usepackage[french]{babel} 
\usepackage{csquotes}
\usepackage[utf8]{inputenc}

% Book headers
\usepackage{fancyhdr}
\pagestyle{fancy}
\fancyhf{}
\fancyhead[L]{\rightmark}
\fancyhead[R]{\thepage}
\renewcommand{\headrulewidth}{0pt}


\definecolor{blueish}{HTML}{CAC8FA}
\definecolor{blueish}{HTML}{CAC8FA}


\newcommand{\comment}[1]{}
\newtheorem{theorem}{Theorem}[section]
\newtheorem{corollary}{Corollary}[theorem]
\newtheorem{proposition}{Proposition}[theorem]
\newtheorem{lemma}[theorem]{Lemma}
\newtheorem{identity}[theorem]{Identity}

\theoremstyle{definition}
\newtheorem{defn}[theorem]{Definition}
\newtheorem{example}[theorem]{Example}
\newenvironment{definition}
  {\vspace{8pt}\begin{mdframed}[backgroundcolor=white,linecolor=blue]\begin{defn}}
  {\end{defn}\end{mdframed}\vspace{4pt}}

\newtheorem{pce}[theorem]{Practice}
\newenvironment{practice}
  {\vspace{8pt}\begin{mdframed}[backgroundcolor=white,linecolor=green]\begin{pce}}
  {\end{pce}\end{mdframed}\vspace{4pt}}


\newtheorem{note}[theorem]{Memo}
\newenvironment{memo}
  {\vspace{8pt}\begin{mdframed}[backgroundcolor=white,linecolor=yellow]\begin{note}}
  {\end{note}\end{mdframed}\vspace{4pt}}



\title{\vspace{-3cm} Geometrization of the Subtle Body}
\author{Luke J Pereira}
\date{}


\begin{document}

\maketitle
\begin{abstract}
\noindent
The grounding of concepts in an embodied intelligence appears to involve domains like intuition, felt-sense, and emotions. It is proposed that these domains are developed from multi-modal sense samples of our perception.
In particular, certain objects stimulate multiple sense doors, like the simultaneous experience of smell, taste, touch while eating a fruit. The synesthetic experience of interest is the intersection and not just the union of different activation sets from different sense doors. Since the encoding of senses varies intrinisically varies because of differences in biological pathways, this can more generally be formulated as invariants of morphisms between encoded sensory samples.
Through various sensory interactions objects in the world, agents develop a spanning structure of these invariant representations of synesthetic perception. This forms a category that can be de-contextualized from particular objects into its own entity of perception.
We can formalize this in representation theory as the irreducible representations that are common between different latent encodings of samples from an object as they are encoded. This encoding can be reduced to as a union of (Lie group) manifolds. More broadly, the collection of invariants can be described as a Skeleton category.
This category of synesthetic invariants has phenomenology comparable to the ``subtle body" in Buddhist writings. The subtle (or energy) body is a non-physical object of meditation that is manipulated during various practices, often producing altered states and generating insights.
This disruption of a spanning body schema is comparable to disruption of GCPRs (mediating various sense perceptions) in psychedelics, which also appears to disrupt predictive coding hierarchies and causes altered states.
Moreover, the refinement of the subtle body and perception itself is at forefront of Buddhist ethics, in contrast to a focus on causality in Western ethics. This leads to the proposal that the described system can be involved in the alignment of artificial intelligence.
\end{abstract}

\newpage

\end{document}
