\documentclass{article}
\usepackage[utf8]{inputenc}
\usepackage[super,square]{natbib}
\usepackage{tabularx}
\usepackage{parskip}
\usepackage[margin=1.4in]{geometry}
\usepackage{csquotes}
\usepackage{mathrsfs}
\usepackage{amsmath}
\usepackage{amsfonts}
\usepackage{amsthm}
\usepackage{amssymb}
\usepackage{hyperref}
\usepackage{graphicx}
\usepackage{float}
\usepackage{mdframed}
\usepackage[dvipsnames]{xcolor}
\usepackage{subcaption}

% Book headers
\usepackage{fancyhdr}
\pagestyle{fancy}
\fancyhf{}
\fancyhead[L]{\rightmark}
\fancyhead[R]{\thepage}
\renewcommand{\headrulewidth}{0pt}


\definecolor{blueish}{HTML}{CAC8FA}

\newcommand{\comment}[1]{}
\newtheorem{theorem}{Theorem}[section]
\newtheorem{corollary}{Corollary}[theorem]
\newtheorem{proposition}{Proposition}[theorem]
\newtheorem{lemma}[theorem]{Lemma}
\newtheorem{identity}[theorem]{Identity}

\theoremstyle{definition}
\newtheorem{defn}[theorem]{Definition}
\newtheorem{example}[theorem]{Example}
\newenvironment{definition}
  {\vspace{8pt}\begin{mdframed}[backgroundcolor=blueish,innertopmargin=4]\begin{defn}}
  {\end{defn}\end{mdframed}\vspace{4pt}}


\title{\vspace{-3cm} Fiber Bundles, Gauges, and Connections}
\author{}
\date{}


\begin{document}
\maketitle
\vspace{-1.5cm}
\tableofcontents
\newpage

\section{Bundles}
\subsection{Fiber Bundles}
    
    
    % \begin{figure}[h]
    % \begin{subfigure}{0.5\textwidth}
    %     \includegraphics[width=8cm]{fiber-bundle.png}
    % \end{subfigure}
    
    % \begin{subfigure}{0.5\textwidth}
    %     \includegraphics[width=5cm]{fiber-bundle-mobius.png}
    % \end{subfigure}
    % \end{figure}
    
    
    
    A fiber bundle makes precise the idea of one topological space (called a fiber) being ``parameterized'' by another topological space (called a base). A fiber bundle also comes with a group action on the fiber. This group action represents the different ways the fiber can be viewed as equivalent. Formally, a fibre bundle is a structure ${\displaystyle (E,\,B,\,\pi ,\,F)}$. The topological space $E$ is known as the \textit{total space} of the fibre bundle, $B$ as the \textit{base space}, and $F$ the standard or template \textit{fiber}. The map ${\displaystyle \pi :E\rightarrow B}$, called the \textit{projection map} or \textit{submersion} of the bundle, is a continuous surjection satisfying a \textit{local triviality condition}. This condition enables a local section of a manifold to be interpreted as a trivial i.e. as cartesian product space ($B\times F$), despite global topology possibly being more complicated, i.e. twisted or non-orientable.
    
    For any $p \in B$, the pre-image ${\displaystyle \pi ^{-1}(\{p\})}$ is homeomorphic to $F$ and is called a \textit{fiber} over $p$. Recall, a homeomorphism is a kind of topological isomorphism, i.e. it is a continuous bijective (invertible) function between topological spaces. Every fibre bundle ${\displaystyle \pi :E\rightarrow B}$ is an open map, since projections of products are open maps. Specifically, we require that for every $p \in B$, there is an open neighborhood ${\displaystyle U\subset B}$ of $p$ (a trivializing neighborhood) such that there is a homeomorphism ${\displaystyle \varphi :\pi ^{-1}(U)\rightarrow U\times F}$ (where ${\displaystyle U\times F}$ is the product space) in such a way that $\pi$ agrees with the projection onto the first factor.  
    
    % Therefore $B$ carries the quotient topology determined by the map $\pi$.
    
    % the fiber at each point of the base space consists of possible coordinate bases for use when describing the values of objects at that poin

    
    The canonical example of a nontrivial bundle $E$ is the Möbius strip. It has the circle that runs lengthwise along the center of the strip as a base $B$ and a line segment for the fiber $F$. A neighborhood $U$ of ${\displaystyle \pi (x)\in B}$ (where $x \in E$) is an arc. The preimage $\pi ^{-1}(U)$ is a partially twisted slice of the strip four squares wide and one long. A homeomorphism $\varphi$ exists that maps the preimage of $U$ to a slice of a cylinder: curved, but not twisted. This pair locally trivializes the strip, with the corresponding trivial bundle $\displaystyle B\times F$ being a cylinder whereas the Möbius strip has an overall twist that is only visible globally.

    A \textit{section} (or cross section) of a fiber bundle $E$ is a continuous right inverse of the projection function $\pi$. In other words, if $E$ is a fiber bundle over a base space, $B$, then a section of that fiber bundle is a continuous map, ${\displaystyle \sigma \colon B\to E}$ such that $\pi (\sigma (x))=x$ for all ${\displaystyle x\in B}$. 
    
    
    Additional structures on $F$ give rise to special types of fiber bundles, e.g. vector bundles or group bundles.  Associated bundles allow derivation of bundles in which the typical fiber of a bundle changes from $F_{1}$ to $F_{2}$, which are both topological spaces with a group action of $G$, e.g. adjoint bundles, frame bundles, determinant bundles, dual bundles. 

\subsection{Vector Bundles}

    If $F = V$ is a vector space, one defines a \textit{vector bundle} with standard fiber $V$ to be a fiber bundle $\pi : E \rightarrow B$ where all fibers $\pi^{-1} (b)$ are vector spaces and the local trivializations $\phi_\alpha$ can be chosen to be fiberwise linear. A homomorphism of two vector bundles is a fiber bundle homomorphism that is fiberwise linear. The fibered product of vector bundles $E_1; E_2$ is a vector bundle (also called Whitney sum and denoted $E1 \oplus E2$).


\subsection{Group Bundle}

    Recall, a Lie group is a group that is also a differentiable manifold where points can be multiplied together, they have inverses, and these operations are defined to be smooth (differentiable). If $F = G$ has the structure of a Lie group, one defines a \textit{group bundle}, $\mathcal G \rightarrow B$ with standard fiber $G$ to be a fiber bundle where all fibers carry group structures and the local trivializations can be chosen to be a fiberwise group homomorphisms. A group bundle homomorphism is a fiber bundle homomorphism which is fiberwise a group homomorphism. The fibered product of group bundles is a group bundle. One has natural bundle maps $\mathcal G \times^B \mathcal G \rightarrow \mathcal G$ (fiberwise group multiplication) and $\mathcal G \rightarrow \mathcal G $ (fiberwise inversion). Similarly, one defines algebra bundles, Lie algebra bundle as well as fiberwise linear actions of group or algebra bundles on vector bundle


\subsection{Principal Bundle}
     A principal $G$-bundle (also called a G-torsor over X) share similar properties to the resulting space of a Cartesian product of a space with a group. They are a fiber bundle $\pi: \mathcal P \rightarrow X$ together with a continuous right action $\mathcal P \times G \rightarrow \mathcal P$ such that $G$ preserves the fibers of $\mathcal P$ (i.e. if $y \in \mathcal P_x$ then $yg \in \mathcal P_x$ for all $g \in G$) and acts freely and transitively (i.e. regularly) on them in such a way that for each $x\in X$ and $y \in \mathcal P_x$, the map $G \rightarrow \mathcal P_x$ sending $g$ to $yg$ is a homeomorphism. In particular each fiber of the bundle is homeomorphic to the group $G$ itself. One can also define principal $G$-bundles in the category of smooth manifolds. Here $\pi : \mathcal P \rightarrow X$ is required to be a smooth map between smooth manifolds, $G$ is required to be a Lie group, and the corresponding action on $\mathcal P$ should be smooth. 

\subsubsection{Associated Bundles}
    Recall, associated bundles allow derivation of bundles in which the typical fiber of a bundle changes from $F_{1}$ to $F_{2}$, which are both topological spaces with a group action of $G$, e.g. adjoint bundles, frame bundles, determinant bundles, dual bundles. 
    
    Let $\pi : \mathcal P \rightarrow B$ be a principal $G$-bundle. Given a
    $G$-manifold $F$, one defines the associated fiber bundle by
    \[
    F (\mathcal P) \equiv \mathcal P \times_G F := (\mathcal P \times F) / G:
    \]
    The space $\mathcal P \times_G F$ is a fiber bundle over $B = \mathcal P=G$ with standard fiber $F$. The sections  $\Gamma^\infty(\mathcal B, \mathcal P \times_G F )$ of this fiber bundle are naturally identified with the space $C^\infty (\mathcal P, F )^G$ of equivariant maps $\mathcal P \rightarrow F$.

\subsubsection{Adjoint Bundle} 

    An adjoint bundle is a vector bundle naturally associated to any principal bundle. The fibers of the adjoint bundle carry a Lie algebra structure making the adjoint bundle into a (nonassociative) algebra bundle. If $V$ is a vector space on which $G$ acts linearly, then $P \times_G V$ is a vector bundle. Taking $V = g$ with the adjoint representation one obtains the adjoint bundle $\mathfrak g(\mathcal P ) := \mathcal P \times_G \mathfrak g$.
    
    If $K$ is a Lie group on which $G$ acts by automorphisms, $\mathcal P \times_G K$ is a group bundle. Taking $K = G$ with $G$ acting by the adjoint action, one obtains a group bundle $G( \mathcal P) := \mathcal P \times_G G$ which is also called the adjoint bundle. It has $\mathfrak g( \mathcal P)$ as its Lie algebra bundle.   

\subsection{Fibrations}
    Fibrations do not necessarily have the local Cartesian product structure that defines the more restricted fiber bundle case, but something weaker that still allows ``sideways'' movement from fiber to fiber. A fibration is like a fiber bundle, except that the fibers need not be the same space, nor even homeomorphic; rather, they are just homotopy equivalent.  Recall, homotopy equivalent implies that if $X$ and $Y$ are a pair of continuous maps $f : X \rightarrow Y$ and $g : Y \rightarrow X$, such that $g \circ f$ is homotopic to the identity map $id_X$ and $f \circ g$ is homotopic to $id_Y$. Intuitively, two spaces $X$ and $Y$ are homotopy equivalent if they can be transformed into one another by bending, shrinking and expanding operations
    
    
    A fibration satisfies an additional condition (the homotopy lifting property) guaranteeing that it will behave like a fiber bundle from the point of view of homotopy theory. Weak fibrations discard even this equivalence for a more technical property. Every vector bundle is a fiber bundle with a fiber homotopy equivalent to a point.  Fibrations are dual to cofibrations, with a correspondingly dual notion of the homotopy extension property; this is loosely known as Eckmann–Hilton duality.
    

\section{Connections}
    The notion of a \textit{connection} defines the idea of transporting data along a curve or family of curves in a parallel and consistent manner. A \textit{covariant derivative} is a linear differential operator which takes the directional derivative of a section of a vector bundle in a covariant manner. It also allows one to formulate a notion of a parallel section of a bundle in the direction of a vector: a \textit{section} $s$ is parallel along a vector $X$ if $\nabla _{X}s=0$. 

\subsection{Ehresmann and Principal Connections}
    An \textit{Ehresmann connection} is a connection in a fibre bundle or a principal bundle made by specifying the allowed directions of motion of the field. Specifically, it singles out a vector subspace of each tangent space to the total space of the fiber bundle, called the horizontal space. A section $s$ is then horizontal (i.e., parallel) in the direction $X$ if $\rm {d}s(X)$ lies in a horizontal space.

    For any fiber bundle $\pi : E \rightarrow B$ the tangent bundle $T E$ of the total space has a distinguished subbundle, the vertical bundle $V E \hookrightarrow T E$. The fiber $V_xE$ for $\pi(x) = b$ is the image of $T_x(F_b)$ under the natural inclusion $T F_b \hookrightarrow T E$. An Ehresmann connection on $E$ is the choice of a complementary horizontal subbundle $HE$ such that $T E = V E \oplus HE$. Equivalently, a connection is a bundle projection $T E \rightarrow V E$ which is left-inverse to the inclusion $V E \rightarrow T E$; one defines $HE$ as the kernel of this projection. 
    
    The Ehresmann connection has the immediate benefit of being definable on a much broader class of structures than vector bundles and is well-defined on a general fiber bundle. Many of the features of the covariant derivative still remain: parallel transport, curvature, and holonomy. With the classical covariant derivatives, covariance is an a posteriori feature of the derivative. However, for an Ehresmann connection, it is possible to impose a generalized covariance principle from the beginning by introducing a Lie group acting on the fibers of the fiber bundle. The appropriate condition is to require that the horizontal spaces be equivariant with respect to the group action.

    If the standard object $F$ has additional structure, one is interested in connections such that parallel transport preserves that structure. For example, if $E$ is a vector bundle, each parallel transport operation, $\Pi^\gamma$, should be a linear map, and for group bundles it should be a fiberwise group homomorphism, and so on.
    An important special case of Ehresmann connections are principal connections on principal bundles, which are required to be equivariant in the principal Lie group action.   A principal connection on a fiber bundle is an equivariant Lie algebra valued 1-form $\theta \in \Gamma^1 (\mathcal P, \mathfrak g)^G$ such that $\iota(\xi_{\mathcal P})\theta = \xi$ (where $\iota$ is an inclusion embedding) for all $\xi \in \mathfrak g$, the Lie algebra.  
    
    The space of principal connections will be denoted $\mathcal A(P)$. The space $\mathcal A(P)$  has a natural affine structure, with underlying vector space the space $\Gamma^1 (B, \mathfrak g( \mathcal P))$ of 1-forms on $B$ with values in the adjoint bundle.

\subsection{Cartan Connections}
    TODO

\section{Gauges}
    
    A gauge can be thought of as a coordinate system that varies depending on one’s location with respect to some base space or parameter space. A gauge transform is a change of coordinates applied to each such location, and a gauge theory is a model for some physical or mathematical system to which gauge transforms can be applied and is typically gauge invariant, in that all physically meaningful quantities are left unchanged or transform naturally under gauge. A principal bundle automorphism is a $G$-equivariant diffeomorphism $\phi : \mathcal P \rightarrow \mathcal P$ taking fibers to fibers. The group of principal bundle automorphisms will be denoted $\hbox{Aut}( \mathcal P)$. 
    
    The space of "coordinate systems" is (non-canonically) identifiable with the isomorphism group $\hbox{Isom}(G)$ of template $G$.  This isomorphism group is called the structure group or gauge group of the class of geometric objects. The gauge group $\hbox{Gau}( \mathcal P) \subseteq \hbox{Aut}(\mathcal P)$ consists of automorphisms $ \phi : \mathcal P \rightarrow \mathcal P$ inducing the identity map on the base $B$. That is, $\hbox{Gau}( \mathcal P)$ is defned by an \textit{exact sequence} of groups
    \[
        1 \longrightarrow \hbox{Gau}( \mathcal P) \longrightarrow \hbox{Aut}( \mathcal P) \longrightarrow \hbox{Diff} B)
    \]
    Let $\theta$ be a principal connection on $\pi : \mathcal P \rightarrow B$. For any path $\gamma : [t_0, t_1] \rightarrow B$, let 
    \[
        \Pi^\theta_\gamma : P_{\gamma(t_0)} \rightarrow P_{(t_1)} 
    \]
    denote parallel transport with respect to $\theta$. For all $\phi \in \hbox{Gau}(\mathcal P)$,
    \[
        \Pi_{\gamma}^{\phi,\theta} = \phi(\gamma(t_1)) \circ  \Pi_{\gamma}^{\theta} \circ  \phi(\gamma(t_0))^{-1}
    \]
    The group of automorphisms $\hbox{Aut}( \mathcal P)$ acts on the space $\mathcal A( \mathcal P)$ of principal connections by pull-back by the inverse. This can be understood as the gauge transformations of connections. We can interpret $\mathcal A( \mathcal P)$ as an infinite dimensional manifold, equipped with an action of an infinite-dimensional Lie Group. 


\section{Moduli Spaces}
    \subsection{Review of differential forms}
    Recall from differential geometry notebook \footnote{\url{https://github.com/lukepereira/notebooks}} the definitions of differential forms, wedge products and the hodge star operations:
    
        \begin{itemize}
            \item A differential \textit{$k$-form }on an open subset $U \subseteq \mathbb R^m$ is an expression of the form 
            \[
                \omega = \sum_{i_1 \dots i_k} \omega i_1\dots i_k dx^{i_1} \wedge \dots \wedge  dx^{i_k}
            \] 
            where $\omega_{i_1\dots i_k} \in C^\infty(U)$ are functions, and the indices are numbers $1 \leq i_1 < \dots < i_k \leq m$. The symbol $\wedge$ denotes the exterior product of two differential forms.
            
            \item The \textit{exterior product} or \textit{wedge product} is the product operator in an exterior algebra. If $\alpha$ and $\beta$ are differential $k$-forms of degrees $p$ and $q$, respectively, then
            \[
                \alpha \wedge \beta=(-1)^{pq} \beta \wedge \alpha. 	
            \]
            It is not (in general) commutative, but it is associative, and bilinear. 
            
            Let $\alpha, \beta \in \Omega^1(M)$. Then we define a wedge product $\alpha \wedge \beta \in \Omega^2 (M)$, as follows:
            \[
                (\alpha \wedge \beta)(X,Y) = \alpha (X)\beta(Y)-\alpha(Y)\beta(X).
            \]
            
            \item Let $V$ be an n-dimensional vector space with basis $\{ e_1, \cdots, e_n \}$ and with unit vector given by $\omega := e_1 \wedge \cdots \wedge e_n$ . Note, the dual of $\omega$ is the volume form, $\hbox{det}$.
            
            An inner product $\langle \cdot,\cdot \rangle$ induces pairs of k-vectors $\alpha, \beta \wedge^k V$ and has the Gram determinant:
            \[
            {\displaystyle \langle \alpha ,\beta \rangle =\det \left(\left\langle \alpha _{i},\beta _{j}\right\rangle \right)_{i,j=1}^{k}}
            \]
            For all pairs of k-vectors, the \textit{Hodge Star operator} can be defined as having property, 
            \[
                \alpha \wedge (* \beta ) = \langle \alpha, \beta \rangle \omega.
            \]
            Applying ${\displaystyle \det }$ to the above equation, we obtain the dual definition:
            \[
                {\displaystyle \det(\alpha \wedge {\star }\beta )=\langle \alpha ,\beta \rangle .}
            \]
            
        \end{itemize}
    

    \subsection{Moduli Space of Connections}
    The quotient space of the space of principal connections and the Gauge  $\mathcal A(P) / \hbox{Gau}(P)$ is called the moduli space of connections. It is still infnite-dimensional. To obtain a finite dimensional moduli spaces, one has to impose additional gauge-invariant constraints on $\theta$: e.g. that it is a flat connection, or more generally a Yang-Mills connection. 
    
    \subsection{Moduli Space of Yang-Mills Connections}
    
    On a principal bundle, we want to choose a canonical connection so that curvature $F_\theta$ vanishes. But not every principal bundle can have a flat connection, and the best one can hope for is that the bundle has curvature as small as possible. On a bundle of connections, a connection is defined by its local forms ${\displaystyle \theta_{\alpha }\in \Omega ^{1}(U_{\alpha },\operatorname {ad} (P))}$. The Yang-Mills action functional $YM(\theta)$ is precisely the square of the ${\displaystyle L^{2}}$-norm of the curvature which has critical points, i.e. local minima, that minimize curvature called Yang-mills connections. These Yang-mills connections, $\mathcal M$, are a finite subset of the moduli space of connections that was originally sought after, i.e. $\mathcal M \subset \mathcal A(P) / \hbox{Gau}(P)$. This is more rigoursly described below. As a side note, in physics the gauge field strength is given by curvature of connections, $F^\theta$, and the energy of the gauge field is the Yang-mills functional.
    
    
    Suppose $\pi : \mathcal P \rightarrow B$ is a principal $G$-bundle over a compact, oriented, Riemannian manifold $B$. The inner product on $T B$ gives rise to an inner product on $T M$ and on all $ \wedge^k T^*M$. Taking the inner product of the differential form, followed by integration over $B$ with respect to the Riemannian volume form, defines an inner product on $\Omega^* (B)$. In terms of the Hodge star operator, 
    \[
        \langle \alpha, \beta \rangle = \int_B \alpha \wedge  * \beta
    \]
    Let $|| \cdot ||$ be the norm corresponding to $\langle \cdot, \cdot \rangle$. The Yang-Mills functional on $\mathcal A(\mathcal P)$ is the functional 
    \[
        \hbox{YM}(\theta) = ||F ||^2 = \int_B (F^\theta, * F^\theta)
    \]
    The Yang-Mills functional is invariant under the action of the gauge group, i.e. $YM(\theta) = YM(\phi,\theta)$, hence all its critical points (called Yang-Mills connections) are invariant as well. A connection $\theta$ is a critical point of the Yang-Mills functional if and only if it satisfies the Yang-Mills equation, $d^\theta * F^\theta = 0$. The quotient of the space of Yang-Mills connections by the action of the gauge group is called the Yang-Mills moduli space. In this context, the term ``moduli'' is used synonymously with ``parameter''. 
    
    
    Moduli of Yang–Mills connections have been most studied when the dimension of the base manifold X is four. Here the Yang–Mills equations admit a simplification from a second-order PDE to a first-order PDE, known as the anti-self-duality equations. The Yang-Mills equations depend upon the Riemannian metric on $B$ only via the star operator on $\Gamma^2 (B)$. The case $\dim B = 4$ is special in that it takes $\Gamma^2 (B)$ to itself, since $4 - 2 = 2$. We mentioned already that in this case the Yang-Mills equations are conformally invariant: Multiplying the metric by a positive function does not change the star operator in middle dimension, hence does not change the Yang-Mills equations. A special type of Yang-Mills connections in $4$ dimensions are those satisfying one of the equations 
    \[
        *F^\theta = F^\theta \text{\ \ or \ \  }  *F^\theta = - F^\theta 
    \]
    (self-duality and anti-self-duality respectively) because for such connections, the Yang-Mills equations are a consequence of the Bianchy identity $d^\theta F^\theta = 0$. A change of orientation of $B$ changes the sign of the  operator, and therefore exchanges the notion of duality and anti-self duality. For certain principal bundles ($\theta$ is a multiple of the second Chern number), ASD connections give the absolute minimum of the Yang-Mills functional.
    
    % Anti-self dual connections over S4 are also called instantons.
    
    % The moduli space for anti-self dual YM-connections for G = SU(2) is the starting point for Donaldson theory of 4-manifolds. As realized by Donaldson, they contain information not only about the topology but also the differentiable structure of 4-manifolds.
    

\section{Examples}

% \subsection{Review: Lagrangian and Hamiltonian}
%     - The Lagrangian L = K - V is big when most of the energy is in kinetic form, and small when most of the energy is in potential form
%     - Lagrangian measures something we could vaguely refer to as the ‘activity’ or ‘liveliness’ of a system: the higher the kinetic energy the more lively the system, the higher the potential energy the less lively. So, we’re being told that nature likes to minimize the total of ‘liveliness’ over time: that is, the total action. (Principle of Least Action)

%     Hamiltonian:
%     - The simplest interpretation of the Hamilton equations is as follows, applying them to a one-dimensional system consisting of one particle of mass m under time-independent boundary conditions: The Hamiltonian H represents the energy of the system (provided that there are NO external forces, or additional energy added to the system), which is the sum of kinetic and potential energy, traditionally denoted T and V, respectively.
%     - e very careful about making the assumption that H = T + V. You should always default to the definition of H which is H = Sum\_i(qdot\_i * partial L/partial qdot\_i) - L
    
\subsection{Relativity}

    In gauge theories, physicists begin with a Lagrangian $L[\phi, \phi']$. The claim is that this $L$ is invariant under the action of some group. What is a bit tricky is that to make this statement a bit more precise requires that we have two fiber bundles at once, namely the principal-$G$ bundle and its associated vector bundle. For each patch of spacetime, $U_i$ where $M$ is base manifold, we pick a map $S:U_i \rightarrow G$ (this will later be the gauge group). Then we pick a certain representation of the group, i.e $\rho :G \rightarrow V$ where $V$ is a vector space. We now define what will later be a section $\psi : U_i \rightarrow[x,\phi]$ where $x$ is a point on the manifold. In this context, gauge invariance means that $[x,\phi] \sim [x,\rho (g^{-1})\phi]$. 
    
    Recall, we have the spacetime patch $U_i$ with the map $S$. With this, we construct the cartesian product $U_i \times G$. If we happen to find two overlapping open sets $U_i$ and $U_j$ then for the sets of points in the intersection we have to make sure things are consistent and so we define functions $t_{ij} : U_i \cap U_j \rightarrow G$ that will act on $G$ i.e $(x,G) \rightarrow (x, t_{ij}(x)G)$. Doing this for the whole manifold $M$ gives us another manifold $P$ that is locally $U_i \times G$. This the principal-G bundle.
    
    Requiring local gauge invariance in the first step meets a snag. The problem involves the map $S$; as we go around on the manifold $M$ we need a method to go from one fiber to another fiber on the principal-G bundle. To do this requires we introduce a connection $\Omega$ on the principal-G bundle. But physicists always work on the base manifold, so we need to pull back $\Omega$ to the base manifold by some section $\sigma$ i.e calculate $\sigma^*\Omega \equiv A$. Since these are locally defined sections, when we are in intersection of two spacetime patches, $U_i \cap U_j$, we will have two sections, $\sigma,\sigma'$. This means we will get $\sigma^*\Omega=A$ and $\sigma'^*\Omega=A'$. The two sections are related by the map $S$ i.e $\sigma'=R_{S(x)}\sigma=\sigma_{S(x)}=\sigma g$. That is, the group acts by a right action.
    
    To calculate $\sigma^*\Omega$ we note that $\langle \sigma^*\Omega, v \rangle = \langle \Omega, \sigma_* v\rangle$ where $v$ is a vector on the principal bundle. A tricky calculation shows $\sigma'_*v = R_{g*}(\sigma_{*}v)+\eta_x(p)$ where $\eta X$ is a fundamental vector field, $X = \langle S^*\theta,v \rangle$ and $\theta$ is the Maurer-Cartan form and $g$ is the image of $S$. To show this works we do the calculation 
    \begin{align*}
            (\sigma'^* \omega)(v) & = <\Omega,\sigma'v> \\
            &= <\Omega,R_{g*}(\sigma_*v)>+<\Omega,\eta_X(pg)\\
            &= <R^*_g\Omega,\sigma_*v> +X \\
            &= <Ad_{g^-1}\Omega,\sigma_*v>+X=<Ad_{g^{-1}}A + S^*\theta,v>
    \end{align*}
    This is the usual transformation rule for the gauge field on the base manifold. We can now state the fact that we have an associated bundle $\mathcal A$ which is $\mathcal{P}\times_G V = (\mathcal{P}\times V)/G$ and is locally $U_i \times V$ with sections as defined in the first step. The sections on this bundle are what physicists call the fields.


\begin{thebibliography}{}

\bibitem[]{}
Tao, T. What is a gauge? (2008).\\ \url{https://terrytao.wordpress.com/2008/09/27/what-is-a-gauge/}
% https://terrytao.wordpress.com/2008/09/27/what-is-a-gauge/

\bibitem[]{}
Eckhard Meinrenken, Principal bundles and connections.\\ \url{http://www.math.toronto.edu/mein/teaching/moduli.pdf}


\bibitem[]{}
Kobayaschi, S. Theory of connections. Annali di Matematica 43, 119–194 (1957).
% file:///home/luke/Downloads/Kobayaschi1957_Article_TheoryOfConnections.pdf

\end{thebibliography}


\end{document}
