\documentclass{article}
\usepackage[utf8]{inputenc}
\usepackage[super,square]{natbib}
\usepackage{tabularx}
\usepackage{parskip}
\usepackage[margin=1.4in]{geometry}
\usepackage{csquotes}
\usepackage{mathrsfs}
\usepackage{amsmath}
\usepackage{amsfonts}
\usepackage{amsthm}
\usepackage{amssymb}
\usepackage{hyperref}
\usepackage{graphicx}
\usepackage{float}
\usepackage{mdframed}
\usepackage[dvipsnames]{xcolor}

% Book headers
\usepackage{fancyhdr}
\pagestyle{fancy}
\fancyhf{}
\fancyhead[L]{\rightmark}
\fancyhead[R]{\thepage}
\renewcommand{\headrulewidth}{0pt}


\definecolor{blueish}{HTML}{CAC8FA}

\newcommand{\comment}[1]{}
\newtheorem{theorem}{Theorem}[section]
\newtheorem{corollary}{Corollary}[theorem]
\newtheorem{proposition}{Proposition}[theorem]
\newtheorem{lemma}[theorem]{Lemma}
\newtheorem{identity}[theorem]{Identity}

\theoremstyle{definition}
\newtheorem{defn}[theorem]{Definition}
\newtheorem{example}[theorem]{Example}
\newenvironment{definition}
  {\vspace{8pt}\begin{mdframed}[backgroundcolor=blueish]\begin{defn}}
  {\end{defn}\end{mdframed}\vspace{4pt}}


\title{\vspace{-3cm} Differential Geometry }
\author{}
\date{}


\begin{document}
\maketitle
\vspace{-1.5cm}
\tableofcontents
\newpage

\section{Prerequisites}


\subsection{Topology of Manifolds Intro}
The fundamental objects of study in differential geometry are manifolds. Roughly, an n-dimensional manifold is a mathematical object that ``locally'' looks like $\mathbb R^n$.  Manifolds in euclidean space are described with a \textit{regular level set}, $S = f^{-1}(a)$ which defines a smooth hypersurface $S \subseteq R^n$. For example, the n-dimensional sphere described by:
\[
    S^n = \{ (x^0, \dots, x^n) \in \mathcal R^{n+1} | (x^0)^2 + \dots + (x^n)^2  = 1\}.
\]
Another example is the 2-Torus, $T^2$. Given real numbers $r, R$ with $0 < r < R$, take a circle of radius $r$ in the $x-z$ plane, with center at $(R,0)$, and rotate about the $z$-axis:
\[
    T^2 = \{ (x,y,z) | (\sqrt{x^2 + y^2} - R)^2 + z^2 + r\}
\]

The sphere, the torus, the double torus, triple torus, and so on are \textit{orientable} surfaces, which essentially means that they have two sides which you might paint in two different colors. It turns out that these are all orientable surfaces, if we consider the surfaces intrinsically and only consider surfaces that are compact in the sense that they don’t go off to infinity and do not have a boundary (thus excluding a cylinder, for example).

Not all surfaces can be realized as 'embedded' in $\mathbb R^3$; for \textit{non-orientable surfaces} one needs to allow for self-intersections. This type of realization is referred to as an immersion: We don’t allow edges or corners, but we do allow that different parts of the surface pass through each other.  An example is the Klein bottle, which is not possible to represent as a regular level set $f^{-1}(0)$ of a function $f$ since any suface has one side where $f$ is positive and another side where $f$ is negative.

The projective plane or projective space is denoted $\mathbb{R} P^2$ and is defined as the set of all lines (i.e., 1-dimensional subspaces) in $\mathbb{R}^3$. we can also think of $\mathbb{R} P^2$ as the set of antipodal (i.e., opposite) points on $S^2$. Splitting the points into those with distance $< \epsilon$ from the equator and those $\geq \epsilon$ produces a Mobius strip and a two-dimensional disc. Generating a smooth curve by gluing the boundary of a Mobius strip to the boundary of a disk is depicted in what's known as Boy’s surface.

Another operation for surfaces, generalizing the procedure of ‘attaching handles’, is the connected sum Given two surfaces $\sigma_1$ and $\sigma_2$, remove small disks around given points $p_1 \in \sigma_1$ and $p_2 \in \sigma_2$, to create two surfaces with boundary circles. Then glue-in a cylinder connecting the two boundary circles, without creating edges. The resulting surface is denoted $\sigma_1\#\sigma_2$.

It turns out that all closed, connected surfaces are obtained from either the 2-sphere $S^2$, the Klein bottle, or  $\mathbb{R} P^2$, by attaching handles with the connected sum.


\subsection{Algebras}

An algebra (over the field $\mathbb R$ of real numbers) is a \textit{vector space} $\mathscr{A}$, together with a \textit{multiplication} (product) $\mathscr{A} \times \mathscr{A} \rightarrow \mathscr{A}$, $(a,b) \mapsto ab$ such that

\begin{enumerate}
    \item The multiplication is associative.
    \item  The multiplication map is linear in both arguments.
\end{enumerate}
The algebra is called commutative if $ab = ba$ for all $a,b \in A$. A unital algebra is an algebra $A$ with a distinguished element $1_{\mathscr{A}} \in \mathscr{A}$ (called the unit), with $1_{\mathscr{A}} a = a = a1_{\mathscr{A}}$ for all $a \in \mathscr{A}$.

One can also consider non-associative product operations on vector spaces, most importantly one has the class of \textit{Lie algebras}. Some examples include: The space of complex numbers which is a unital, commutative algebra. $\mathbb H \cong \mathbb R^4$ of quaternions which is a unital, non-commutative algebra. The space of $n\times n$ matricies which is a noncommutative unital algebra. Given a topological space $X$, one has the algebra $C(X)$ of continuous $\mathbb R$-valued functions.

A \textit{homomorphism} of algebras $\phi : \mathscr{A} \rightarrow \mathscr{A}'$ is a linear map preserving products: $\phi(ab) = \phi(a)\phi(b)$. It is called an \textit{isomorphism} of algebras if $\phi$ is invertible. For the special case $\mathscr{A}' = A$ , these are also called algebra \textit{automorphisms} of $\mathscr{A}$. Note that the algebra automorphisms form a group under composition.

\subsection{Derivations}
\begin{definition}
A derivation of an algebra $\mathscr A$ is a linear map $D : \mathscr A \rightarrow \mathscr A$ satisfying the product rule 
\[
    D(a_1 a_2) = D(a_1)a_2 +a_1D(a_2).
\].
\end{definition}

If $dim A < \infty$, a derivation is an infinitesimal automorphism of an algebra.

Any given $x \in A$ defines a derivation $D(a) = [x,a] := xa-ax$. These are called inner derivations. If $A$ is commutative (for example $A = C ^\infty(M)$) the inner derivations are all trivial. At the other extreme, for the matrix algebra $A = \text{Mat}\mathbb R(n)$, one may show that every derivation is inner.

If $A$ is a unital algebra, with unit $1_A$, then $D(1A) = 0$ for all derivations $D$.

Given two derivations $D_1,D_2$ of an algebra $A$, their commutator $[D_1,D_2] = D_1 D_2 - D_2 D_1$ is again a derivation.

If the algebra $A$ is commutative, then the space of derivations is a 'left-module over $A$'. That is, if $D$ is a derivation and $x \in A$ then $a \mapsto (xD)(a) := xD(a)$ is again a derivation

\subsection{Equivalence Relations}
[Omitted]

\subsection{On relation to Physics}
In Albert Einstein’s theory of General Relativity from 1916, space-time was regarded as a 4-dimensional 'curved' manifold with no distinguished coordinates. A local observer may want to introduce local $xyz$ coordinates to perform measurements, but all physically meaningful quantities must admit formulations that are coordinate-free. At the same time, it would seem unnatural to try to embed the 4-dimensional curved space-time continuum into some higher-dimensional flat space, in the absence of any physical significance for the additional dimensions. Some years later, gauge theory once again emphasized coordinate-free formulations, and provided physics motivations for more elaborate constructions such as fiber bundles and connections. There are many subbranches of differential geometry, for example complex geometry, Riemannian geometry, or symplectic geometry, which further subdivide into sub-sub-branches.

\section{Manifolds}
\subsection{Atlases and Charts}

Manifolds will initially be described in intrinsic or manifestly  coordinate-free terms. The basic feature of manifolds is the existence of 'local coordinates'. The transition from one set of coordinates to another should be smooth.
\begin{definition} \textbf{(Smoothness and diffeomophisms)}

Let $U \subseteq \mathbb R^m$ and $V \subseteq \mathbb R^n$ be open subsets. A map $F : U \rightarrow V$ is called \textit{smooth} if it is infinitely differentiable. The set of smooth functions from $U$ to $V$ is denoted $C^{\infty}(U,V)$. The map $F$ is called a \textit{diffeomorphism} from $U$ to $V$ if it is invertible, and the inverse map $F^{-1}: V \rightarrow U$ is again smooth.
\end{definition}


\begin{definition} \textbf{(Jacobian matrix)}

For a smooth map $F \in C^{\infty}(U,V)$ between open subsets $U \subseteq R^m$ and $V \subseteq R^n$, and any $x \in U$, one defines the Jacobian matrix $DF(x)$ to be the $n \times m$ matrix of partial derivatives,
\[
    (DF(x))^i_j = \frac{\partial F^i}{\partial x^j}
\]
Its determinant is called the Jacobian matrix of $F$ at $x$.
\end{definition}

\begin{theorem} \textbf{(Inverse function theorem)}

The inverse function theorem states that $F$ is a diffeomorphism if and only if it is invertible, and for all $x \in U$, the Jacobian matrix $DF(x)$ is invertible. This means one does not actually have to check smoothness of the inverse map.
\end{theorem}



\begin{definition} \textbf{(Charts)}

Let $M$ be a set.
\begin{enumerate}
    \item An $m$-dimensional (coordinate) chart $(U,\phi)$ on $M$ is a subset $U \subseteq M$ together with a map $\phi : U \rightarrow R^m$ , such that $\phi(U) \subseteq R^m$ is open and $\phi$ is a bijection from $U$ to $\phi(U)$.
    
    \item  Two charts $(U,\phi)$ and $(V,\psi)$ are called compatible if the subsets $\phi(U \cap V)$ and $\psi(U \cap V)$ are open, and the transition map
    \[
        \psi \circ \phi^{-1}  = \phi(U \cap V) \rightarrow \psi(U \cap V)
    \]
    is a diffeomorphism known as a change of coordinates. As a special case, charts with $U\cap V = \emptyset $ are always compatible.
\end{enumerate}
\end{definition}

\begin{definition} \textbf{(Atlas)}

Let $M$ be a set. An $m$-dimensional atlas on $M$ is a collection of coordinate charts $\mathscr{A} =  \{(U_\alpha, \phi_\alpha) \}$ such that,
\begin{enumerate}
    \item The $U_\alpha$ cover all of $M$, i.e. $\bigcup_\alpha U_\alpha = M$.
    \item For all indicies $\alpha, \beta$, the charts $(U_\alpha, \phi_\alpha)$ and $(U_\beta, \phi_\beta)$ are compatible.
\end{enumerate}

\end{definition}

\begin{definition} \textbf{(Sterographic projection)}

Regard $\mathbb R^2$ as the coordinate subspace of $\mathbb R^3$ on which $z = 0$ runs through the center of the sphere; the ``equator'' is the intersection of the sphere with this plane.
Let $N = (0, 0, 1)$ be the ``north pole'', and let $M$ be the rest of the sphere. For any point $P$ on $M$, there is a unique line through $N$ and $P$, and this line intersects the plane $z = 0$ in exactly one point $P'$. Define the stereographic projection of $P$ to be this point $P'$ in the plane. 
\end{definition}

\begin{example} \textbf{(Atlas on the 2-sphere)}

Let $S^2 \subseteq \mathbb R^3$ be the unit sphere. Let $n = (0,0,1)$ be the north pole, and $s =(0,0,-1)$ be the south pole, and define an atlas with two charts $(U_+,\phi_+)$ and $(U_-,\phi_-)$, where
\[
U_+ = S^2 - {s}, U_- = S^2 - {n}
\]
and the stereographic projection from the south pole is given by,
\begin{align*}
    \phi_+ : U_+ \rightarrow \mathbb R^2, p \mapsto \phi_+(p)\\
    \phi_+(x, y,z) = \bigg( \frac{x}{1+z},\frac{y}{1+z} \bigg)
\end{align*}
and the stereographic projection from the north pole is given by,
\begin{align*}
    \phi_- : U_+ \rightarrow \mathbb R^2, p \mapsto \phi_-(p)\\
    \phi_-(x, y,z) = \bigg( \frac{x}{1-z},\frac{y}{1-z} \bigg).
\end{align*}
The transition map on the overlap of the two charts is,
\begin{equation*}
    \phi_- \circ \phi_+^{-1} (u,v) =  \bigg( \frac{u}{u^2 + v^2}, \frac{v}{u^2 + v^2} \bigg).
\end{equation*}

\end{example}

% \begin{example} \textbf{(Affine lines in $\mathbb R^2)$}

% Omitted
% \end{example}

The 2-sphere with the atlas given by stereographic projections onto the $x-y$-plane, and the 2-sphere with the atlas given by stereographic projections onto the $y-z$-plane, should be one and the same manifold $S^2$. To resolve this, we will use the following notion of compatibility.

\begin{definition} \textbf{(Compatibility)}

Suppose $\mathscr{A} =  \{(U_\alpha, \phi_\alpha) \}$ is an $m$-dimensional atlas on $M$, and let $(U,\phi)$ be another chart. Then $(U,\phi)$ is said to be \textit{compatible} with $\mathscr{A}$ if it is compatible with all charts $(U_\alpha, \phi_\alpha)$ of $\mathscr{A}$
\end{definition}

Note that  $(U,\phi)$ is compatible with the atlas $\mathscr{A} =  \{(U_\alpha, \phi_\alpha) \}$ if and only if the union $\mathscr{A} \cup {(U,\phi)}$ is again an atlas on $M$. This suggests defining a bigger atlas, by using all charts that are compatible with the given atlas with the following lemma.

\begin{lemma}

Let $\mathscr{A} =  \{(U_\alpha, \phi_\alpha) \}$ be a given atlas on the set $M$. If two charts $(U,\phi), (V,\psi)$ are compatible with $\mathscr{A}$, then they are also compatible with each other.
\end{lemma}

\begin{theorem}

Given an atlas $\mathscr{A} =  \{(U_\alpha, \phi_\alpha) \}$ on $M$, let $\Tilde{\mathscr{A}}$ be the collection of all charts $(U,\phi)$ that are compatible with $\mathscr{A}$. Then $\Tilde{\mathscr{A}}$ is itself an atlas on $M$, containing $\mathscr{A}$. In fact, $\Tilde{\mathscr{A}}$ is the largest atlas containing $\mathscr{A}$.
\end{theorem}

\begin{definition} \textbf{(Maximal and equivalent atlases)}

An atlas $\mathscr{A}$ is called \textit{maximal} if it is not properly contained in any larger atlas. Given an arbitrary atlas $\mathscr{A}$, one calls $\Tilde{\mathscr{A}}$ the maximal
atlas determined by $\mathscr{A}$.  Two atlases are called equivalent if every chart of one atlas is compatible with every chart in the other atlas. Any maximal atlas determines an equivalence class of atlases, and vice versa.
\end{definition}


\subsection{Definition of manifold}

\begin{definition}
An $m$-dimensional manifold is a set $M$, together with a maximal atlas $\mathscr{A} =  \{(U_\alpha, \phi_\alpha) \}$ with the following properties: 
\begin{enumerate}
    \item \textbf{(Countability condition)} $M$ is covered by countably many coordinate charts in $A$. That is, there are indices $\alpha_1, \alpha_2, \dots$ with
    \[
        M = \bigcup_i U_{\alpha_i}.
    \]
    \item \textbf{(Hausdorff condition)} For any two distinct points $p,q \in M$ there are coordinate charts $(U_\alpha, \phi_\alpha)$ and $(U_\beta, \phi_\beta)$ in $A$ such that $p \in U_\alpha, q \in U_\beta$, with
    \[
        U_\alpha \cap U_\beta = \emptyset.
    \]
\end{enumerate}
\end{definition}

The charts $(U,\phi) \in A$ are called (coordinate) charts on the manifold $M$. The countability condition is used for various arguments involving a proof by induction. The Hausdorff condition rules out some strange examples that don’t quite fit the idea of a space that is locally like $\mathbb R^n$.

\begin{lemma}

Let $M$ be a set with a maximal atlas $\mathscr{A} =  \{(U_\alpha, \phi_\alpha) \}$, and suppose $p,q \in M$ are distinct points contained in a single coordinate chart $(U,\phi) \in \mathscr{A}$. Then we can find indices $\alpha,\beta$ such that $p \in U_\alpha, q \in U_\beta$, with $U_\alpha \cap U_\beta = \emptyset$.
\end{lemma}

\subsection{Examples of manifolds}
\subsubsection{N-spheres}
The construction of an atlas for the 2-sphere $S^2$, by stereographic projection, also works for the n-sphere
\[
    S^n = \{ (x^0, \dots, x^n) \in \mathcal R^{n+1} | (x^0)^2 + \dots + (x^n)^2  = 1\}.
\]
Let $U_{\pm}$ be the subsets obtained by removing $(\pm 1, 0,\dots, 0)$. Stereographic projection defines bijections $\phi_{\pm} : U_{\pm} \rightarrow \mathbb R^{n}$,  where $\phi_{\pm} (x^0, x^1, \cdots, x^n) = (u^1, \cdots, u^n)$ with $\displaystyle u^{i} = \frac{x^j}{1\pm x^0}.$

Writing $u = (u^1,\dots,u^n)$, the transition functions are given by,
\[
    (\phi_- \circ \phi_+^{-1})(u) = \frac{u}{||u||^2}.
\]

\subsubsection{Products and n-torus}

Given manifolds $M,M'$ of dimensions $m,m'$, with atlases $\{(U_\alpha,\phi_\alpha) \}$ and $\{(U_\beta,\phi_\beta)\}$, the cartesian product $M \times M'$ is a manifold of dimension $m + m'$. An atlas is given by the product charts $U_\alpha \times U_\beta$ with the product maps $\phi_\alpha \times \phi_\beta' : (x, x') \mapsto (\phi_\alpha(x), \phi_\beta'(x'))$. For example, the 2-torus $T^2 = S^1 \times S^1$ becomes a manifold in this way, and likewise for the n-torus, $T^n = S^1 \times \dots \times S^1$.

\subsubsection{Real projective spaces}

The n-dimensional projective space $\mathbb R P^n$, is the set of all lines $l \subseteq \mathbb R^{n+1}$. It may also be regarded as a quotient space,
\[
\mathbb R P^n  = (\mathbb R^{n+1} \setminus \{0\}) / \sim
\]
$\mathbb R P^n$ has a standard atlas, $\mathscr{A} =  {(U_0,\phi_0),\dots,(U)n,\phi_n)}$ defined as follows. For $j = 0,\dots,n,$ let $U_j = \{(x^0 : \dots : x^n) \in \mathbb R P^n | x^j \neq 0\}$ be the set for which the j-th coordinate is non-zero, and put 
\[
\phi_j : U_j \rightarrow \mathbb R^n, (x^0 : \dots : x^n ) \mapsto \bigg ( \frac{x^0}{x^j} ,\dots, \frac{x^n}{x^j} \bigg).
\]

Geometrically, viewing $\mathbb RP^n$ as the set of lines in $\mathbb R^{n+1}$, the subset $U_j \subseteq \mathbb RP^n$ consists of those lines $l$ which intersect the affine hyperplane $H_j = \{x \in \mathbb R^{n+1} | x^j = 1\}$ and the map $\phi_j$ takes such a line $l$ to its unique point of intersection $l \cap H_j$, followed
by the identification $H_j \cong  \mathbb R^n$ (dropping the coordinate $x^j = 1$). In low dimensions, we have that $\mathbb RP^0$ is just a point, while $\mathbb RP^1$is a circle.

\subsubsection{Complex projective spaces}

Similar to the real projective space, one can define a complex projective space $\mathbb C P^n$ as the set of complex 1-dimensional subspaces of $\mathbb C^{n+1}$.
\[
    \mathbb C P^n = (\mathbb C^{n+1} \setminus \{0\})/ \sim
\]
Alternatively, letting $S^{2n+1} \subseteq \mathbb C^{n+1} = \mathbb R^{2n+2}$ be the 'unit sphere' consisting of complex vectors of length $||z|| = 1$, we have
\[
 \mathbb C P^n = S^{2n+1} / \sim
\]
One defines charts $(U_j ,\phi_j)$ similar to those for the real projective space:
\begin{align*}
    U_j &= \{(z^0 : \dots : z^n) | z^j \neq 0\} \\
    \phi_j : U_j \rightarrow \mathbb C^{2n},& \ \ (z^0 : \dots : z^n ) \mapsto \bigg ( \frac{z^0}{z^j} ,\dots, \frac{z^n}{z^j} \bigg).
\end{align*}
The transition maps between charts are given by similar formulas as for $\mathbb RP^n$ (just replace $x$ with $z$).  The transition maps are not only smooth but even \textit{holomorphic}, making $\mathbb C P^n$ an example of a complex manifold (of complex dimension $n$).

\subsubsection{Grassmannians}
The set $Gr(k,n)$ of all $k$-dimensional subspaces of $\mathbb R^n$ is called the Grassmannian of $k$-planes in $\mathbb R^n$. As a special case, $Gr(1,n) = \mathbb RP^{n-1}$. We will show that for general $k$, the Grassmannian is a manifold of dimension $dim(Gr(k,n)) = k(n-k)$. [Omitted]

\subsubsection{Complex Grassmannians}

Similar to the case of projective spaces, one can also consider the complex Grassmannian $Gr \mathbb C(k,n)$ of complex $k$-dimensional subspaces of $\mathbb C^n$ . It is a manifold of dimension $2k(n-k)$, which can also be regarded as a complex manifold of complex dimension $k(n-k)$.

\subsection{Oriented manifolds}

The notion of an orientation on a manifold will become crucial later, since integration of differential forms over manifolds is only defined if the manifold is oriented.

\begin{definition} \textbf{(Oriented atlas and manfold)}

The compatibility condition between charts $(U,\phi),(V,\psi)$ on a set $M$ is that the change of coordinates map $\phi \circ \psi^{-1})$ is a diffeomorphism. In particular, the Jacobian matrix $D(\phi \circ \psi^{-1}))$ of the transition map is invertible, and hence has non-zero determinant.  If the determinant is $> 0$  everywhere, then we say $(U,\phi),(V,\psi)$ are \textit{oriented-compatible}.  An \textit{oriented atlas} on $M$ is an atlas such that any two of its charts are oriented-compatible; a \textit{maximal oriented atlas} is one that contains every chart that is oriented-compatible with all charts in this atlas. An \textit{oriented manifold} is a set with a maximal oriented atlas, satisfying the Hausdorff and countability conditions. A manifold is called orientable if it admits an oriented atlas.

\end{definition}

The spheres $S^n$ are orientable. To see this, consider the atlas with the two charts given by stereographic projections. The Jacobian matrix $D(\phi_-, \psi^{-1})_+)(u)$ has determinant of $-||u||^{-2n}$ which  is not an oriented atlas. To remedy this, simply compose one of the charts, with the map $(u_1,u_2,\dots,u_n) \mapsto (-u_1,u_2,\dots,u_n)$; then with the resulting new coordinate map $\Tilde{\phi}_-$ the atlas $(U_+,\phi_+),(U_-,\Tilde{\phi}_-)$ will be an oriented atlas.

One can show that the real projective space $\mathbb R P^n$ is orientable if and only if $n$ is odd or $n = 0$. More generally, the Grassmannians space $Gr(k,n)$ is orientable if and only if $n$ is even or $n = 1$. The complex projective spaces $\mathbb C P^n$ and complex Grassmannians $Gr \mathbb C(k,n)$ are all orientable. This
follows because the transition maps for their standard charts, as maps between open subsets of $\mathbb C^m$, are actually complex-holomorphic, and this implies that as real maps, their Jacobian has positive determinant.


\subsection{Open subsets}

Let $M$ be a set equipped with an $m$-dimensional maximal atlas $\mathscr{A} =  \{(U_\alpha, \phi_\alpha) \}$.

\begin{definition} \textbf{(Open subset)}

A subset $U \subseteq M$ is open if and only if for all charts $(U_\alpha, \phi_\alpha) \in \mathscr{A}$ the set $\phi_\alpha(U  \cap U_\alpha)$ is open.
\end{definition}

\begin{proposition}

Given $U \subseteq M$, let $ \mathscr{B} \subseteq \mathscr{A}$ be any collection of charts whose union contains $U$. Then $U$ is open if and only if for all charts $(U_\beta,\phi_\beta)$ from $\mathscr{B}$, the sets $\phi_{\beta}(U \cap U_\beta)$ are open.
\end{proposition}

This means that to check that a subset $U$ is open, it is not actually necessary to verify this condition for all charts. As the above proposition shows, it is enough to check for any collection of charts whose union contains $U$. In particular, we may take $\mathscr{A}$ to be any atlas, not necessarily a maximal atlas.



If $\mathscr{A}$  is an atlas on $M$, and $U \subseteq M$ is open, then $U$ inherits an atlas by restriction:
\[
    \mathscr{A}  = \{ (U \cap U_\alpha, \phi_\alpha |_{U \cap U_\alpha } ) \}
\]

\begin{proposition}
An open subset of a manifold is again a manifold.
\end{proposition}

\begin{proposition}
Let $M$ be a set with an $m$-dimensional maximal atlas. The collection of all open subsets of M has the following properties
\begin{enumerate}
    \item $\emptyset, M$ are open.
    \item The intersection $U \cap U'$ of any two open sets $U,U'$ is again open.
    \item The union $\cap_i U_i$ of an arbitrary collection $U_i, i \in I$ of open sets is again open.
\end{enumerate}
\end{proposition}


These properties mean, by definition, that the collection of open subsets of $M$ define a \textit{topology} on $M$. This allows us to adopt various notions from topology:

\begin{enumerate}
    \item A subset $A \subseteq M$ is called \textit{closed} if its complement $M \setminus A$ is open.
    \item $M$ is called \textit{connected} if the only subsets $A \subseteq M$ that are both closed and open are $A = \emptyset$ and $A = M$.
    \item If $U$ is an open subset and $p \in U$, then $U$ is called an open neighborhood of $p$. More generally, if $A \subseteq U$ is a subset contained in $M$, then $U$ is called an \textit{open neighborhood} of $A$.
\end{enumerate}

The Hausdorff condition in the definition of manifolds can now be restated as the condition that any two distinct points $p,q \in M$ have disjoint open neighborhoods. (It is not necessary to take them to be domains of coordinate charts.) It is immediate from the definition that domains of coordinate charts are open. Indeed, this gives an alternative way of defining the open sets.

\subsection{Compact subsets}
Another important concept from topology that we will need is the notion of \textit{compactness}.

\begin{definition} \textbf{(Compactness)}

A subset $A \subseteq \mathbb R^m$ is compact if it has the following property: For every collection ${U_\alpha}$ of open subsets of $R^m$ whose union contains $A$, the set $A$ is already covered by finitely many subsets from that collection. 
\end{definition}
In short, $A \subseteq M$ is compact if every open cover admits a finite subcover.

\begin{theorem} \textbf{(Heine-Borel)}

A subset $A \subseteq R^m$ is compact if and only if it is closed
and bounded.
\end{theorem}

\begin{proposition}

If $A \subseteq M$ is contained in the domain of a coordinate chart $(U,\phi)$, then $A$ is compact in $M$ if and only if $\phi(A)$ is compact in $R^n$.
\end{proposition}

The proposition is useful, since we can check compactness of $\phi(A)$ by using the Heine-Borel criterion. For more general subsets of $M$, we can often decide compactness by combining this result with the following:

\begin{proposition}

If $A_1,\dots,A_k \subseteq M$ is a finite collection of compact subsets, then their union $A = A_1 \cap \dots \cap A_k$ is again compact.
\end{proposition}

A simpler way of verifying compactness is by showing that they are closed and bounded subsets of $\mathbb R^N$ for a suitable $N$.

\begin{proposition}

Let $M$ be a set with a maximal atlas. If $A \subseteq M$ is compact, and $C \subseteq M$ is closed, then $A \cap C$ is compact.
\end{proposition}

\begin{proposition}
If $M$ is a manifold, then every compact subset $A \subseteq M$ is closed.
\end{proposition}

\newpage
\section{Smooth Maps}
\subsection{Smooth functions on manifolds}
The notion of smooth functions on open subsets of Euclidean spaces
carries over to manifolds: A function is smooth if its expression in local coordinates is smooth,

\begin{definition} \textbf{(Smoothness)}

A function $f : M \rightarrow R$ on a manifold $M$ is called smooth if for all charts $(U,\phi)$ the function
\[
    f \circ\phi^{-1}: \phi(U) \rightarrow \mathbb{R}
\]
is smooth. The set of smooth functions on $M$ is denoted $C^{\infty}(M)$.
\end{definition}

Since transition maps are diffeomorphisms, it suffices to check the condition for the charts from any given atlas which need not be the maximal atlas.

Given an open subset $U \subseteq M$, we say that a function $f$ is smooth on $U$ if its restriction $f |_U$ is smooth. (Here we are using that $U$ itself is a manifold.) Given $p \in M$, we say that $f$ is smooth at $p$ if it is smooth on some open neighborhood
of $p$.

\begin{lemma}
Smooth functions $f \in C^{\infty}(M)$ are continuous: For every open subset $J \subseteq \mathbb R$, the pre-image $f^{-1}(J) \subseteq M$ is open.
\end{lemma}


From the properties of smooth functions on $\mathbb R^m$, one immediately gets the following properties of smooth functions on manifolds $M$:
\begin{enumerate}
    \item If $f,g \in C^{\infty}(M)$ and $\lambda, \mu \in \mathbb R$, then $\lambda f + \mu g \in C^\infty(M)$.
    \item If $f,g \in C^\infty(M)$, then $f g \in C^\infty(M)$.
    \item $1 \in C^{\infty}(M)$ (where $1$ denotes the constant function $p \mapsto 1$)
\end{enumerate}

These properties say that $C^{\infty}(M)$ is an \textit{algebra} with unit 1. 

\begin{proposition}
Suppose $M$ is any set with a maximal atlas, and $p \neq q$ are two points in $M$. Then the following are equivalent:
\begin{enumerate}
    \item There are open subsets $U,V \subseteq M$ with $p \in U, q \in V, U \cap V = \emptyset$
    \item There exists $f \in C^\infty(M)$ with $f(p) \neq f(q)$.
\end{enumerate}
\end{proposition}

\begin{corollary} \textbf{(Criterion for Haudorff condition)}


A set $M$ with an atlas satisfies the Hausdorff condition if and only if for any two distinct points $p,q \in M$, there exists a smooth function $f \in C^\infty(M)$ with $f(p) \neq f(q)$. In particular, if there exists a smooth injective map $F : M \rightarrow \mathbb R^N$, then $M$ is Hausdorff.
\end{corollary}

\subsection{Smooth maps between manifolds}
\begin{definition} 

A map $F : M \rightarrow N$ between manifolds is smooth at $p \in M$ if there are coordinate charts $(U,\phi)$ around $p$ and $(V,\psi)$ around $F(p)$ such that $F(U) \subseteq V$ and such that the composition
\[
    \psi \circ F \circ \phi^{-1}: \phi(U) \rightarrow \psi(V)
\]
is smooth. The function $F$ is called a smooth map from $M$ to $N$ if it is smooth at all $p \in M$.

\end{definition}

The condition for smoothness at p does not depend on the choice
of charts. To check smoothness of $F$, it suffices to take any atlas of M with the property that $F(U_\alpha) \subseteq V_\alpha$ and then check smoothness of the maps. Smooth maps $M \rightarrow R$ are the same thing as smooth functions on $M$, $C^{\infty} (M,R) = C^{\infty}(M)$.

\begin{proposition}
Suppose $F_1 : M_1 \rightarrow M_2$ and $F_2 : M_2 \rightarrow M_3$ are smooth maps. Then the composition $ F_2 \circ F_1 : M_1 \rightarrow M_3$ is smooth.
\end{proposition}

\subsection{Diffeomorphisms of manifolds}

\begin{definition} \textbf{(Diffeomorphic manifolds)}

A smooth map $F : M \rightarrow N$ is called a \textit{diffeomorphism} if it is invertible, with a smooth inverse $F^{-1} : N \rightarrow M$. Manifolds $M,N$ are called diffeomorphic if there exists a diffeomorphism from $M$ to $N$.
\end{definition}

In other words, a diffeomorphism of manifolds is a bijection of the underlying sets that identifies the maximal atlases of the manifolds. Manifolds that are diffeomorphic are therefore considered 'the same manifolds'.

\begin{definition} \textbf{(Homeomorphic manifolds)}
A continuous map $F : M \rightarrow N$ is called a \textit{homeomorphism} if it is invertible, with a continuous inverse.  Manifolds $M,N$ are called homeomorphic if there exists a homeomorphism from $M$ to $N$.
\end{definition}

Manifolds that are homeomorphic are considered 'the same topologically'. Since every smooth map is continuous, every diffeomorphism is a homeomorphism.

\begin{example}
The standard example of a homeomorphism of smooth manifolds that is not a diffeomorphism is the map $\mathbb R \rightarrow \mathbb R, x \mapsto x^3$. Indeed, this map is smooth and invertible, but the inverse map $y \mapsto y ^{\frac{1}{3}}$ is not smooth.
\end{example}

It is quite possible for two manifolds to be homeomorphic but not diffeomorphic, these are known as \textit{exotic manifolds}. An \textit{exotic sphere} is homeomorphic but not diffeomorphic to the standard Euclidean n-sphere. It is known that there are no exotic manifold structures on $\mathbb R^n$ for $\mathbb R^n$ with $n \neq 4$, where there are uncountably many such.

\subsection{Examples of smooth maps}

\subsubsection{Products, diagonal maps}
\begin{enumerate}
    \item  If $M,N$ are manifolds, then the projection maps are smooth. Take product charts  $U_\alpha \times V_\beta$.
    \[
        p^{T}_M : M \times N  \rightarrow M, \ \
        p^{T}_N :  M \times N \rightarrow N
    \]
    
    \item The diagonal inclusion is smooth. In a coordinate chart around a point, the map is the restriction to a subset of the diagonal inclusion.
    \[
        \Delta_M : M \rightarrow  M \times M
    \]
    
    \item  Suppose $F : M \rightarrow N$ and $F' : M' \rightarrow N'$ are smooth maps. Then the direct product is smooth.
    \[
        F \times F' : M \times M' \rightarrow N \times N'
    \]
    
\end{enumerate}

\subsubsection{The diffeomorphism \texorpdfstring{$\mathbb R P^1 \cong S^1$}{RP1 ∼= S1} }

We have seen that $\mathbb R P^1 \cong S^1$. To obtain a diffeomorphism, we construct a bijection between the standard atlases of both of the manifolds described previously

\subsubsection{The diffeomorphism \texorpdfstring{$\mathbb C P^1 \cong S^2$}{CP1 ∼= S2}}

By a similar reasoning, we find $\mathbb C P^1 \cong S^2$. For $S^2$ we use the atlas given by stereographic projection.  Regarding $u$ as a complex number the normis just the absolute value of $u$, and the transition map becomes $u \mapsto \frac{1}{u}$. Note that it is not quite the same as the transition map for the standard atlas of $\mathbb C P^1$,
which is given by $u \mapsto u^{-1}$. We obtain a unique diffeomorphism such that $\phi_+ \circ F \circ \phi_0^{-1}$ is the identity and $\phi_- \circ F \circ \phi_1^{-1}$ is complex conjugation

\subsubsection{Maps to and from projective space}

The quotient map $\pi$ is smooth, as one verifies by checking in the standard atlas for $\mathbb R P^n$.
\[
    \pi : \mathbb R^{n+1} \setminus \{0\} \rightarrow \mathbb R P^N, \ \ 
    x = (x^0, \dots, x^n) \mapsto (x^0, \dots, x^n)
\]
Given a map $F : \mathbb R P^n \rightarrow N$ to a manifold $N$, let $\Tilde{F} = F \circ \pi : \mathbb R ^{n+1}\setminus \{0\} \rightarrow N$ be its composition with the projection map $\pi : \mathbb R ^{n+1}\setminus \{0\} \rightarrow \mathbb R P^n$. That is, $\Tilde{F} (x^0 , \dots, x^n ) = F(x^0 : \dots : x^n )$.

We claim that the map $F$ is smooth if and only the corresponding map $\Tilde{F}$ is smooth. One direction is clear: If $F$ is smooth, then $\Tilde{F} = F \circ \pi$ is a composition of smooth maps. For the other direction, assuming that $\Tilde{F}$ is smooth, note that for the standard chart $(U_j ,\phi_j)$, and the maps
\[
    (F \circ \phi^{-1}_j )(u^1 ,\dots, u^n ) = \Tilde{F}(u^1 ,\dots, u^i ,1,u ^{i+1} ,\dots,u^n ),
\] are smooth. An analogous argument applies to the complex projective space $\mathbb C P^n$ , taking the $x^i$ to be complex numbers $z^i$
 

\subsubsection{The Hopf fibration, a.k.a. the quotient map \texorpdfstring{$S^{2n+1} \rightarrow \mathbb C P^n$}{S2n+1 to CPn}}

As mentioned above, quotient map $q : C^{n+1} \setminus \{0\} \rightarrow \mathbb C P^n$ is smooth.  Since any class $[z] = (z 0 : \dots : z^n )$ has a representative with $|z^0|^2 + \dots +|z^n|^2 = 1$, and $|z^i|^2 = (x^i)^2 + (y^i)^2$ for $z^i = x^i + \sqrt{-1}y^i$, we may also regard $\mathbb C P^n$ as a set of equivalence classes in the unit sphere $S^{2n+1} \subseteq \mathbb R^{2n+2} = \mathbb C^{n+1}$. The resulting quotient map
\[
    \pi : S^{2n+1} \rightarrow \mathbb C P^n 
\]
is again smooth, because it can be written as a composition of two smooth maps $\pi = q \circ \tau$  where $\tau : S ^{2n+1} \mapsto \mathbb R^{2n+2} \setminus \{0\} = \mathbb C^{n+1} \setminus \{0\}$ is the inclusion map.

For any $p \in \mathbb C P^n$ , the corresponding fiber $\pi^{-1} (p) \subset S^{2n+1}$ is diffeomorphic to a circle $S^1$ (which we may regard as complex numbers of absolute value 1). Indeed, given any point $(z^0 , \dots ,z^n ) \in \pi^{-1} (p)$ in the fiber, the other points are obtained as $(\lambda z^0 ,\dots, \lambda z^n )$ where $|\lambda| = 1$.

In other words, we can think of
\[
    S^{2n+1} = \bigcup_{p \in \mathbb C P^n} \pi^{- 1}(p)
\]
as a union of circles, parametrized by the points of $\mathbb C P^n$. This is an example of a \textit{fiber bundle} or \textit{fibration}. 

An import important case occurs when $n = 1$. Identifying $\mathbb C P^1 \cong  S^2$ as above, the map $\pi$ becomes a smooth map $\pi : S^3 \rightarrow S^2$ with fibers diffeomorphic to $S^1$. This map appears in many contexts; it is called the
\textit{Hopf fibration}.

Let $S \in S^3$ be the 'south pole', and $N \in S^3$ the 'north pole'. We have that $S^3 - \{S\} \cong R^3$ by stereographic projection. The set $\pi^{-1} (\pi(S)) -  \{S\}$ projects to a straight line (think of it as a circle with 'infinite radius'). The fiber $\pi ^{-1} (N)$ is a circle that goes around the straight line. If $Z \subseteq S^2$ is a circle at a given 'latitude', then $\pi^{-1} (Z)$ is is a 2-torus. For $Z$ close to north pole $N$ this 2-torus is very thin, while for $Z$ approaching the south pole $S$ the radius goes to infinity. Each such 2-torus is itself a union of circles $\pi^{-1} (p), p \in Z$. Those circles are neither the usual 'vertical' or 'horizontal' circles of a 2-torus in $\mathbb R^3$ , but instead are 'tilted'. In fact, each such circle is a 'perfect geometric circle' obtained as the intersection of its 2-torus with a carefully positioned affine 2-plane. Moreover, any two of the circles $\pi^{-1} (p)$ are 'linked' as though they were in a chain.


A calculation shows that over the charts $U_+,U_-$ from stereographic projection, the Hopf fibration is just a product. That is, one has
\[
    \pi^{-1} (U_+) \cong U_+ \times S^1, \ \ \pi^{-1} (U_-) \cong U_- \times S^1
\]
In particular, the pre-image of the closed upper hemisphere is a solid 2-torus $D^2 \times S^1$ (with $D^2 = \{ z \in \mathbb C| |z| \leq 1 \}$ the unit disk), geometrically depicted as a 2-torus in $\mathbb R^3$ together with its interior. We hence see that the $S^3$ may be obtained by gluing two solid 2-tori along their boundaries $S^1 \times S^1$.

\subsection{Submanifolds}
\begin{definition}
A subset $S \subseteq M$ is called a \textit{submanifold} of dimension $k \leq m$, if for all $p \in S$ there exists a coordinate chart $(U,\phi)$ around $p$ such that
\[
    \phi(U \cap S) = \phi(U) \cap \mathbb R^k .
\] 
Charts $(U,\phi)$ of $M$ with this property are called submanifold charts for $S$.
\end{definition}

\begin{definition}
A chart $(U, \phi)$ such that $U \cap S = \emptyset$ and $\phi (U) \cap \mathbb R^k = \emptyset$ is considered a \textit{submanifold chart}. The existence of submanifold charts is only required for points $p$ that lie in $S$. 
\end{definition}

Strictly speaking, a submanifold chart for $S$ is not a chart for $S$, but is a chart for $M$ which is adapted to $S$. Submanifold charts restrict to charts for $S$, and this may be used to construct an atlas for $S$.

\begin{proposition}
Suppose $S$ is a submanifold of $M$. Then $S$ is a $k$-dimensional manifold in its own right, with atlas consisting of all charts $(U \cap S, \phi|U \cap S)$ such that $(U,\phi)$ is a submanifold chart.
\end{proposition}

\begin{example} (Open subsets). The $m$-dimensional submanifolds of an $m$-dimensional manifold are exactly the open subsets.
\end{example}

\begin{example} (Projective spaces). For $k < n$, regard $\mathbb R P^k \subseteq \mathbb R P^n$ as the subset of all  $(x^0 : \dots : x^n )$ for which $x^{k+1} = \dots = x^n = 0$. These are submanifolds, with the standard charts $(U_i ,\phi_i)$ for $\mathbb R P^n$ as submanifold charts. Similarly, $\mathbb C P^k \subseteq \mathbb C P^n$ are submanifolds, and for $n < n'$ we have $Gr(k,n) \subseteq Gr(k,n')$ as a submanifold.

\end{example}

\begin{example} (Spheres). For $k < n$, regard $S^k \subseteq S^n$ as the subset where the last $n-k$ coordinates are zero. These are submanifolds where the charts for $S^n$ given by stereographic projection are submanifold charts.

\end{example}

\begin{proposition}
Let $F : M \rightarrow N$ be a smooth map between manifolds of dimensions $m$ and $n$. Then 
\[
    graph(F) = \{(F(p), p) | p \in M \} \subseteq N \times M 
\]
is a submanifold of $N \times M$, of dimension equal to the dimension of $M$.

This result has the following consequence: If a subset of a manifold, $S \subseteq M$, can be locally described as the graph of a smooth map, then $S$ is a submanifold.
\end{proposition}

\begin{proposition}
The inclusion map $i : S \rightarrow M, p \mapsto p,$ which takes any point of $S$ to the same point but viewed as a point of $M$, is smooth.
\end{proposition}

\begin{proposition}
Suppose $S$ is a submanifold of $M$. Then the open subsets of $S$ for its manifold structure are exactly those of the form $U \cap S$, where $U$ is an open subset of $M$.
\end{proposition}

In other words, the topology of $S$ as a manifold coincides with the 'subspace topology' as a subset of the manifold $M$.

As a consequence, if a manifold $M$ can be realized realized as a submanifold $M \subseteq \mathbb R^n$, then $M$ is compact with respect to its manifold topology if and only if it is compact as a subset of $R^n$, if and only if it is a closed and bounded subset of $\mathbb R^n$.

\subsection{Smooth maps of maximal rank}

Let $F \in C^\infty(M,N)$ be a smooth map. Then the fibers (level sets) $F^{-1}(q) = \{x \in M| F(x) = q\}$ for $q \in N$ need not be submanifolds, in general.

\subsubsection{The rank of a smooth map}
Let $U \subseteq \mathbb R^m$ and $V \subseteq \mathbb R^n$ be open subsets, and $F \in  C^\infty(U,V)$ a smooth map.

\begin{definition}
The \textit{derivative} of $F$ at $p \in U$ is the linear map
\[
    D_pF : \mathbb R^m \rightarrow \mathbb R^n, \ \ v \mapsto \frac{d}{dt}\Bigr|_{\substack{ t=0 }} F(p+tv).
\]

Recall that the \textit{rank} of a linear map is the dimension of its range. The rank of $F$ at $p$ is the rank of this linear map:
\[
    \text{rank}_p(F) = \text{rank}(D_p F).
\]
\end{definition}

Equivalently, $D_p F$ is the $n \times m$ matrix of partial derivatives and the rank of $F$ at $p$ is the rank of this matrix i.e., the number of linearly independent rows or the number of linearly independent columns.

\begin{definition}
    Let $F \in C^\infty(M,N)$ be a smooth map between manifolds, and $p \in M$. The rank of $F$ at $p \in M$ is defined as
    \[
        \text{rank}_p(F) = \text{rank}_{\phi(p)}(\psi \circ F \circ \phi^{-1}).
    \]
    for any two coordinate charts $(U,\phi)$ around $p$ and $(V,\psi)$ around $F(p)$ such that $F(U) \subseteq V$.
\end{definition}


\begin{definition}
    The map $F$ is said to have \textit{maximal rank} at $p$ if
    \[
        \text{rank}_p(F) = \text{min}(\text{dim} M, \text{dim} N).
    \]
    A point $p \in M$ is called a \textit{critical point} for $F$ if $\text{rank}_p(F) < \text{min}(\text{dim} M, \text{dim} N)$.
\end{definition}

\subsubsection{Local diffeomorphisms}

\begin{theorem} (Inverse Function Theorem for $\mathbb R^m$).

Let $F \in C^\infty(U,V)$ be a smooth map between open subsets of $\mathbb R^m$, and suppose that the derivative $D_pF$ at $p \in U$ is invertible. Then there exists an open neighborhood $U_1 \subseteq U$ of $p$ such that $F$ restricts to a diffeomorphism $U_1 \rightarrow F(U_1)$.
\end{theorem}

The theorem tells us that for a smooth bijection, a sufficient condition for smoothness of the inverse map is that the differential (i.e., the first derivative) is invertible everywhere.

\begin{theorem} (Inverse function theorem for manifolds)

Let $F \in C^\infty(M,N)$ be a smooth map between manifolds of the same dimension $m = n$. If $p \in M$ is such that $\text{rank}_p(F) = m$, then there exists an open neighborhood $U \subseteq M$ of $p$ such that $F$ restricts to a diffeomorphism $U \rightarrow F(U)$.
\end{theorem}


A smooth map $F \in C^\infty(M,N)$ is called a local diffeomorphism if $\text{dim} M = \text{dim}N$, and $F$ has maximal rank everywhere. By the theorem, this is equivalent to the condition that every point $p$ has an open neighborhood $U$ such that $F$ restricts to a diffeomorphism $U \rightarrow F(U)$.

\subsubsection{Level sets, submersions}
\begin{proposition}
Suppose $F \in C^\infty(U,V)$ is a smooth map between open subsets $U \subseteq R^m$ and $V \subseteq R^n$, and suppose $p \in U$ is such that the derivative $D_p F$ is surjective. Then there exists an open neighborhood $U_1 \subseteq U$ of $p$ and a diffeomorphism $\kappa : U_1 \rightarrow \kappa (U_1) \subseteq \mathbb R^m$ such that
\[
   (F \circ \kappa - 1)(u^1, \dots, u^m) = (u^{m-n+1},\dots,u^m)
\]

for all $u = (u^1 ,\dots,u^m) \in \kappa(U_1)$.
\end{proposition}

Again, this result has a version for manifolds:

\begin{theorem}
Let $F \in C ^\infty(M,N)$ be a smooth map between manifolds of dimensions $m \geq n$, and suppose $p \in M$ is such that $\text{rank}_p(F) = n$. Then there exist coordinate charts $(U,\phi)$ around $p$ and $(V,\psi)$ around $F(p)$, with $F(U) \subseteq V$, such that 
\[
    (\psi \circ F \circ \phi^{-1} )(u' ,u'') = u'' 
\]
for all $u = (u',u'') \in \phi(U)$. In particular, for all $q \in V$ the intersection $F^{-1} (q)\cap U$ is a submanifold of dimension $m-n$.

\end{theorem}

\begin{definition}
    Let $F \in C^\infty(M,N)$. A point $q \in N$ is called a \textit{regular value} of $F \in C^\infty(M,N)$ if for all $x \in F^{-1} (q)$, one has $\text{rank}_x(F) = \text{dim}N$. It is called a \textit{singular value} if it is not a regular value.
\end{definition}

Note that regular values are only possible if $\text{dim}N \leq \text{dim}M$. Note also that all points of $N$ that are not in the image of the map F are considered regular values. We may
restate the theorem as follows:
\begin{theorem}
For any regular value $q \in N$ of a smooth map $F \in C^\infty(M,N)$, the level set $S = F^{-1}(q)$ is a submanifold of dimension $\text{dim}S = \text{dim}M - \text{dim}N$.
\end{theorem}

\begin{definition}
    A smooth map $F \in C^\infty(M,N)$ is a \textit{submersion} if $\text{rank}_p(F) = \text{dim}N$ for all $p \in M$.
    
    Thus, for a submersion all level sets $F^{-1} (q)$ are submanifolds.
\end{definition}

\begin{example}
Recall that $\mathbb CP^n$ can be regarded as a quotient of $S^{2n+1}$. Using charts, one can check that the quotient map $\pi : S^{2n+1} \rightarrow \mathbb C P^n$ is a submersion. Hence its fibers $\pi^-1 (q)$ are 1-dimensional submanifolds. As discussed before these fibers are circles. As a special case, the Hopf fibration $S^3 \rightarrow S^2$ is a submersion.
\end{example}

\begin{example}

Let $\mathbb H = \mathbb C^2 = \mathbb R^4$ be the \textit{quaternionic numbers}. The unit quaternions are a 3-sphere $S^3$ . Generalizing the definition of $\mathbb R P^n$ and $\mathbb C P^n$, there are also quaternionic projective spaces, $\mathbb H P^n$. These are quotients of the unit sphere inside $\mathbb H^{n+1}$ , hence one obtains submersions 
\[
    S^{4n+3} \mapsto \mathbb H P^n;
\]
the fibers of this submersion are diffeomorphic to $S^3$. For $n = 1$, one can show that $\mathbb HP^1 = S^4$, hence one obtains a submersion $\pi : S^7 \rightarrow S^4$ with fibers diffeomorphic to $S^3$.
\end{example}


\subsubsection{Example: The Steiner surface}

[Omitted]

\subsubsection{Immersions}

\begin{proposition}
Suppose $F \in C^\infty(U,V)$ is a smooth map between open subsets $U \subseteq \mathbb R^m$ and $V \subseteq \mathbb R^n$, and suppose $p \in U$ is such that the derivative $D_p F$ is injective. Then there exist smaller neighborhoods $U_1 \subseteq U$ of $p$ and $V_1 \subseteq V$ of $F(p)$, with $F(U_1) \subseteq V_1$, and a diffeomorphism $\chi : V_1 \rightarrow \chi(V_1)$, such that $(\chi \circ F)(u) = (u,0) \in R^m \times R^{n-m}$
\end{proposition}

The manifolds version reads as follows: 
\begin{theorem}
Let $F \in C^\infty(M,N)$ be a smooth map between manifolds of dimensions $m \infty n$, and $p \in M$ a point with $\text{rank}_p(F) = m$. Then there are coordinate charts $(U,\phi)$ around $p$ and $(V,\psi)$ around $F(p)$ such that $F(U) \subseteq V$ and 
\[
    (\psi \circ F \circ \phi{-1})(u) = (u,0).
\]

In particular, $F(U) \subseteq N$ is a submanifold of dimension $m$.
\end{theorem}

\begin{definition}
A smooth map $F : M \mapsto N$ is an immersion if $\text{rank}_p(F) = \text{dim}M$ for all $p \in M$.    
\end{definition}

\begin{theorem}
If $M$ is a compact manifold, then every injective immersion $F : M \rightarrow N$ is an embedding as a submanifold $S = F(M)$. By an embedding, we will mean an immersion given as the inclusion map for a submanifold.
\end{theorem}

\newpage
\section{The Tangent Bundle}
For embedded submanifolds $M \subseteq \mathbb R^n$, the tangent space $T_p M$ at $p \in M$ can be defined as the set of all velocity vectors $v = \gamma(0$), where $\gamma : J \rightarrow M$ is a smooth curve with $\gamma(0) = p$; here $J \subseteq R$ is an open interval around $0$. It turns out that $T_pM$ becomes a vector subspace of $\mathbb R^n$. For a general manifold, we will define $T_pM$ as a set of directional derivatives.

\begin{definition} (Tangent spaces - first version)

Let $M$ be a manifold, $p \in M$. The \textit{tangent space} $T_pM$ is the set of all linear maps $v : C^{\infty}(M) \rightarrow \mathbb R$ of the form 
\[
    v(f) = \frac{d}{dt}\Bigr|_{\substack{ t=0 }}  f(\gamma(t))
\]
for some smooth curve $\gamma \in C^\infty(J,M)$ with $\gamma(0) = p$.

The elements $v \in T_pM$ are called the \textit{tangent vectors} to $M$ at $p$.
\end{definition}

The following local coordinate description makes it clear that $T_pM$ is a linear subspace of the vector space $L(C^\infty(M),R)$ of linear maps $C^\infty(M) \rightarrow R$, of dimension equal to the dimension of $M$.

\begin{theorem}
Let $(U,\phi)$ be a coordinate chart around $p$. A linear map $v : C^\infty(M) \rightarrow \mathbb R$ is in $T_pM$ if and only if it has the form, 
\[
    v(f) = m \sum_{i=1}^m a^i \frac{\partial (f \circ \phi^{-1} )}{ \partial u^i}\Bigr|_{u = \phi(p)}
\]
for some $a = (a^1 ,\dots,a^m) \in \mathbb R^m$
\end{theorem}

We can use this result as an alternative definition of the tangent space, namely:

\begin{definition} (Tangent spaces - second version)


Let $(U,\phi)$ be a chart around $p$. The tangent space $T_pM$ is the set of all linear maps $v : C^\infty(M) \rightarrow \mathbb R$ of the form 
\[
     v(f) = m \sum_{i=1}^m a^i \frac{\partial (f \circ \phi^{-1} )}{ \partial u^i}\Bigr|_{u = \phi(p)} 
\]
for some $a = (a^1 ,\dots,a^m) \in \mathbb R^m$.
\end{definition}

It is not immediately obvious from this second definition that $T_pM$ is independent of the choice of coordinate chart, but this follows from the equivalence with the first definition. Any choice of coordinate chart $(U,\phi)$ around $p$ defines a vector space isomorphism $T_pM \cong \mathbb R^m$, taking $v$ to $a = (a^1 ,\dots,a^m)$. In particular, we see that if $U \subseteq \mathbb R^m$ is an open subset, and $p \in U$, then $T_pU$ is the subspace of the space of linear maps $C^\infty(M) \rightarrow \mathbb R$ spanned by the partial derivatives at $p$.

We now describe yet another approach to tangent spaces which again characterizes ''directional derivatives'' in a coordinate-free way, but without reference to curves $\gamma$. Note first that every tangent vector satisfies the product rule, also called the Leibniz rule:
\begin{lemma} (Leibniz rule)

Let $v \in T_pM$ be a tangent vector at $p \in M$. Then 
\[ 
    v(f g) = f(p) v(g) +v(f)g(p) 
\]
for all $f,g \in C ^\infty(M)$.
\end{lemma}
Alternatively, in local coordinates it is just the product rule for partial derivatives. It turns out that the product rule completely characterizes tangent vector:

\begin{theorem}

A linear map $v : C^\infty(M) \rightarrow \mathbb R$ defines an element of $T_pM$ if and only if it satisfies the Leibniz product rule.
\end{theorem}

\begin{definition} (Tangent spaces - third version)

The tangent space $T_pM$ is the space of linear maps $C^\infty(M) \rightarrow \mathbb R$ satisfying the product rule, 
\[
    v(f g) = f(p)v(g) +v(f)g(p)
\]
for all $f,g \in C^\infty(M)$.
\end{definition}

\begin{definition} (Tangent Vectors)

The \textit{velocity vectors} of curves are elements of the tangent space. Let $J \subseteq \mathbb R$ be an open interval, and $\gamma \in C^\infty(J,M)$ a smooth curve. Then for any $t_0 \in J$, the tangent (or velocity) vector $\dot{\gamma}(t_0) \in T_{\gamma(t_0)}M$ at time $t_0$ is given in terms of its action on functions by $(\dot{\gamma}(t_0))(f) = \frac{d}{dt} \Bigr |_{t=t_0} f(\gamma(t))$. We will also use the notation $\frac{d\gamma}{dt} \Bigr|_(t_0)$ or $\frac{d\gamma}{dt} \Bigr |_{t_0}$ to denote the velocity vector.
\end{definition}

\subsection{Tangent map}
\subsubsection{Definition of the tangent map, basic properties}

The following definition generalizes the derivative to smooth maps between manifolds.

\begin{definition}

Let $M,N$ be manifolds and $F \in C^\infty(M,N)$. For any $p \in M$, we define the tangent map to be the linear map 
\[
    T_pF : T_pM \rightarrow T_{F(p)}N
\]
given by 
\[
    (T_{p}F(v))(g) = v(g \circ F) 
\]
for $v \in T_pM$ and $g \in C^\infty(N)$

\end{definition}

\begin{proposition}
If $v \in T_p M$ is represented by a curve $\gamma : J \rightarrow M$, then $(T_pF)(v)$ is represented by the curve $F\circ \gamma$.
\end{proposition}

\begin{definition}(Pull-backs, push-forwards)

For smooth maps $F \in C^\infty(M,N)$, one can consider various 'pull-backs' of objects on $N$ to objects on $M$, and 'push-forwards' of objects on $M$ to objects on $N$. Pull-backs are generally denoted by $F^{*}$, push-forwards by $F_{*}$. For example, functions on $N$ pull back, curves push forward on $M$, and tangent vectors to $M$ also push forward.
\end{definition}


\begin{proposition} (Chain rule)
Let $M,N,Q$ be manifolds. Under composition of maps $F \in C^\infty(M,N)$ and $F0 \in C^\infty(N,Q)$,
\[
    T_p(F'\circ F) = T_{F(p)} F' \circ T_pF. 
\]

\end{proposition}

\subsubsection{Coordinate description of the tangent map}
\begin{proposition}

Let $F \in C^\infty(U,V)$ is a smooth map between open subsets $U \subseteq \mathbb R^m $ and $V \subseteq \mathbb R^n$. For all $p \in M$, the tangent map $T_pF$ is just the derivative (i.e., Jacobian matrix) $D_pF$ of $F$ at $p$.
\end{proposition}

Now that we have recognized $TpF$ as the derivative expressed in a coordinate-free way, we may liberate some of our earlier definitions from coordinates:
\begin{itemize}
    \item  The rank of $F$ at $p \in M$, denoted $\text{rank}p(F)$, is the rank of the linear map $T_pF$.
    \item $F$ has maximal rank at $p$ if $\text{rank}p(F) = min(dim M, dim N)$.
    \item $F$ is a submersion if $T_pF$ is surjective for all $p \in M$,
    \item $F$ is an immersion if $T_pF$ is injective for all $p \in M$,
    \item $F$ is a local diffeomorphism if $T_pF$ is an isomorphism for all $p \in M$.
    \item $p \in M$ is a critical point of $F$ is $T_pF$ does not have maximal rank at $p$.
    \item $q \in N$ is a regular value of $F$ if $T_pF$ is surjective for all $p \in F^{-1} (q)$ (in particular, if $q \not\in F(M))$
    \item $q \in N$ is a singular value if it is not a regular value.
\end{itemize}

\subsection{Tangent spaces of submanifolds}

Suppose $S \subseteq M$ is a submanifold, and $p \in S$. Then the tangent space $T_pS$ is canonically identified as a subspace of $T_pM$. 
Indeed, since the inclusion $i : S \mapsto M$ is an immersion, the tangent map is an injective linear map, $T_pi : T_pS \rightarrow T_pM$, and we identify $T_pS$ with the subspace given as the image of this map.

Recall, the kernel of a linear mapping, also known as the null space or nullspace, is the set of vectors in the domain of the mapping which are mapped to the zero vector.

\begin{proposition}

Let $F \in C ^{\infty}(M,N)$ be a smooth map, having $q \in N$ as a regular value, and let $S^\infty F^{-1} (q)$. For all $p \in S$,
\[
    T_pS = \ker(T_pF),
\]
as subspaces of $T_pM$.

\end{proposition}

\begin{corollary}

Suppose $V \subseteq \mathbb R^n$ is open, and $q \in \mathbb R^k$ is a regular value of $F \in C^\infty(M, \mathbb R^k)$, defining an embedded submanifold $M = F^{-1} (q)$. For all $p \in M$, the tangent space $T_pM \subseteq T_p \mathbb R^n = \mathbb R^n$ is given as 
\[
    T_p M = \ker(T_pF) \equiv \ker(D_pF).
\]
\end{corollary}

\begin{example}

Various matrix Lie Groups are submanifolds $G \subseteq \text{Mat}_{\mathbb R}(n)$, consisting of invertible matrices with the properties
\[
    A,B \in G \implies AB \in G, A \in G \implies A^{-1} \in G.
\]

The tangent space to the identity (group unit) for such matrix Lie groups $G$ turns out to be important; it is commonly denoted by lower case Fraktur letters $\mathfrak {g} = T_I G$.

\begin{enumerate}
    \item The \textit{matrix Lie group} 
    \[
        GL(n,R) = \{A \in \text{Mat} \mathbb R(n) | \det(A) \neq 0 \}
    \]
    of all invertible matrices is an open subset of $\text{Mat} \mathbb R(n)$, hence 
    \[
        \mathscr{gl}(n,R) = \text{Mat} \mathbb R(n)
    \]
    is the entire space of matrices.
    
    \item For the group $O(n)$, consisting of matrices with $F(A) := A ^T A = I$, we have computed $T_A F(X) = X^T A + AX ^T$. For $A = I$, the kernel of this map is 
    \[\mathfrak{o}(n) = \{ X \in \text{Mat}_\mathbb R(n) | X = -X \}.\]
    
    \item For the \textit{special linear group} $SL(n,R) = \{A \in \text{MatR}(n)| \det(A) = 1\}$, given as the level set $F^{-1}(1)$ of the function $\det : \text{Mat}_\mathbb R(n) \rightarrow R$, we calculate 
    \[
        D_A F(X) = \frac{d}{dt} \Bigr |_{t=0} F(A+tX) =  \frac{d}{dt} \Bigr |_{t=0} \det(A+tX) = \frac{d}{dt} \Bigr |_{t=0} \det(I+tA^{-1}X) = tr(A^{-1} X),
    \]
    where $tr : \text{Mat}_\mathbb R(n) \rightarrow R$ is the trace (sum of diagonal entries). Hence 
    \[ \mathfrak{sl}(n,R) = \{X \in \text{Mat}_\mathbb R (n)| tr(X) = 0 \}.\]
\end{enumerate}

\end{example}

\subsubsection{Example: Steiner’s surface revisited}

[Omitted]

\subsubsection{The tangent bundle}

\begin{proposition}
For any manifold $M$ of dimension $m$, the tangent bundle
\[
    TM = \bigsqcup_{p\in M} T_p M
\]
(disjoint union of vector spaces) is a manifold of dimension $2m$. The map 
\[
    \pi : TM \rightarrow M
\]
taking $v \in TpM$ to the base point $p$, is a smooth submersion, with fibers in the tangent spaces.

\end{proposition}

\begin{proposition}

For any smooth map $F \in C^\infty(M,N)$, the map $T F : TM \rightarrow TN$ given on $T_pM$ as the tangent maps $T_pF : T_pM \rightarrow T_{F(p)}N$, is a smooth map
\end{proposition}

\newpage 
\section{Vector Fields}

\subsection{Vector fields as derivations}
\begin{definition} (Vector Fields - first definition).

A collection of tangent vectors $X_p, p \in M$ defines a vector field $X \in \mathfrak X \in M$ if and only if for all functions $f \in C^\infty(M)$ the function $p \mapsto X_p(f)$ is smooth. The space of all vector fields on $M$ is denoted $\mathfrak X(M)$. We hence obtain a linear map $X : C^\infty(M) \rightarrow C^\infty(M)$ such that 
\[
    X (f)|_p = X_p(f).
\]
\end{definition}
Since each $X_p$ satisfy the product rule (at $p$), it follows that $X$ itself satisfies a product rule. We can use this as an alternative definition:

\begin{definition} (Vector Fields - second definition).

A vector field on $M$ is a linear map $X : C^\infty (M) \rightarrow C^\infty (M)$ satisfying the product rule, 
\[
    X(f g) = X(f)g + f X(g)
\]
for $f,g \in C^\infty(M)$.
\end{definition}

We can also express the smoothness of the tangent vectors $X_p$ in terms of coordinate charts $(U,\phi)$. Recall that for any $p \in U$, and all $f \in C^\infty(M)$, the tangent vector $X_p$ is expressed as
\[
    X_p(f) = \sum_{i=1}^m a^i (u) \frac{\partial}{\partial u^i} \Bigr |_{u=\phi(p)}  (f \circ \phi^{-1})
\]

\begin{proposition}

The collection of tangent vectors $X_p,\  p \in M$ define a vector field if and only if for all charts $(U,\phi)$, the functions $a^i : \phi(U) \rightarrow \mathbb R$ defined by 
\[
    X_{\phi^{-1}(u)} (f) = \sum_{i=1}^m a^i (u) \frac{\partial}{\partial u^i} (f \circ \phi^{-1}),
\]
are smooth.

\end{proposition}

\subsection{Vector fields as sections of the tangent bundle}


\begin{definition} (Vector fields – third definition).

A vector field on $M$ is a smooth map $X \in C^\infty(M,TM)$ such that $\pi \circ X$ is the identity.

\end{definition}

It is common practice to use the same symbol X both as a linear map from smooth functions to smooth functions, i.e. $X : M \rightarrow TM$, or as a map into the tangent bundle, $X : C^\infty(M) \rightarrow C^\infty(M)$. The latter case can also be expressed as the 'Lie derivative' to avoid confusion: $L_X : C^\infty(M) \rightarrow C^\infty(M)$

\subsection{Lie brackets}
\begin{theorem}

For any two vector fields $X,Y \in \mathfrak X(M)$ (regarded as derivations), the commutator 
\[
    [X,Y] := X \circ Y - Y \circ X : C^\infty (M) \rightarrow C^\infty (M)
\]
is again a vector field.

\end{theorem}

\begin{definition} (Lie Brackets)

The vector field $[X,Y] := X \circ Y - Y \circ X$ is called the \textit{Lie bracket} of $X,Y \in \mathfrak X(M)$.

\end{definition}

Note: When calculating Lie brackets $X \circ Y - Y \circ X$ of vector fields $X,Y$ in local coordinates, it is not necessary to work out the second order derivatives – we know in advance that these are going to cancel out.

Let $S \subseteq M$ be a submanifold. A vector field $X \in \mathfrak X(M)$ is called tangent to $S$ if for all $p \in S$, the tangent vector $X_p$ lies in $T_pS \subseteq T_pM$. (Thus $X$ restricts to a vector field $X|_S \in \mathfrak X(S)$.)

\begin{proposition}
If two vector fields $X,Y \in \mathfrak X(M)$ are tangent to a submanifold $S \subseteq M$, then their Lie bracket is again tangent to $S$.
\end{proposition}

\subsection{Related vector fields}

\begin{definition} (F-related)

Let $F \in C^\infty(M,N)$ be a smooth map. Vector fields $X \in \mathfrak X(M)$ and $Y \in \mathfrak X(N)$ are called $F$-related, written as $X \sim_F Y$, if $T_pF(X_p) = Y_{F(p)}$ for all $p \in M$.

\end{definition}

\begin{example}

If $F$ is a diffeomorphism, then $X \sim_F Y$ if and only if $Y = F_* X$. In particular, if $N = M$, then an equation $X \sim_F X$ means that $X$ is invariant under $F$.

\end{example}

The $F$-relation of vector fields also has a simple interpretation in terms of the 'differential operator' picture.

\begin{proposition}
One has $X \sim_F Y$ if and only if for all $g \in C^\infty(N), \ X(g \circ F) = Y(g) \circ F$.

\end{proposition}

\begin{theorem}

Let $F \in C ^\infty(M,N)$ For vector fields $X_1,X_2 \in \mathfrak X(M)$ and $Y_1,Y_2 \in \mathfrak X(M)$, we have 
\[
    X_1 \sim_F Y_1, \ X_2 \sim_F Y_2 \rightarrow [X1,X2] \sim_F [Y1,Y2].
\]
\end{theorem}

\subsection{Flows of vector fields}
Recall, For any curve $\gamma : J \rightarrow M$, with $J \subseteq \mathbb R$ an open interval, and any $t \in J$, the velocity vector $\dot{\gamma}(t) \equiv \frac{d\gamma}{dt} \in T_{\gamma(t)}M$ is defined as the tangent vector, given in terms of its action on functions as $(\dot{\gamma}(t))(f) = \frac{d}{dt} f(\gamma(t))$. (The dot signifies a t-derivative.)

Equivalently, one may think of the velocity vector as the image of $\frac{\partial}{\partial t} |_{t} \in T_tJ \cong \mathbb R$ under the tangent map $T_t \gamma : \dot{\gamma}(t) = (T_t\gamma)( \frac{\partial}{\partial t}|_t)$.

\begin{definition}

Suppose $X \in \mathfrak X(M)$ is a vector field on a manifold $M$. A smooth curve $\gamma \in C^\infty(J, M)$, where $J \subseteq R$ is an open interval, is called a solution curve to $X$ if $\dot{\gamma}(t) = X_{\gamma(t)}$ for all $t \in J$.

\end{definition}


Geometrically, this means that at any given time $t$, the value of $X$ at $\gamma(t)$ agrees with the velocity vector to $\gamma$ at $t$, i.e. $\frac{\partial}{\partial t} \sim_\gamma X$.

\begin{example}
Consider first the case that $M = U \subseteq \mathbb R^m$. Here curves $\gamma (t)$ are of the form,
\[
    \gamma(t) = x(t) = (x^1 (t),\dots, x^m (t)),
\]
hence,
\[
    \dot{\gamma}(t) = \sum_{i=1}^m \frac{dx^i}{dt} \frac{\partial}{\partial x^i} \Bigr |_{x(t)}.
\]
On the other hand, the vector field has the form $X = \sum_{i=1}^m a^i (x) \frac{\partial}{\partial x i}$. This becomes the system of first order ordinary differential equations, 
\[
    \frac{dx^i}{dt} = a^i (x(t)), i = 1,\dots, m.
\]
\end{example}

\begin{theorem} (Existence and uniqueness theorem for ODE's)

Let $U \subseteq \mathbb R$ m be an open subset, and $a \in C^\infty(U, \mathbb R^m)$. For any given $x_0 \in U$, there is an open interval $J_{x_0} \subseteq \mathbb R$ around $0$, and a solution $x : J_{x_0} \rightarrow U$ of the ODE
\[
    \frac{d x^i}{dt} = a^i (x(t)), i = 1, \dots, m
\]
with initial condition $x(0) = x_0$, and which is maximal in the sense that any other solution to this initial value problem is obtained by restriction to some subinterval of $J_{x0}$.

\end{theorem}

Thus, $J_{x_0}$ is the maximal open interval on which the solution is defined.

\begin{theorem}  (Dependence on initial conditions for ODE's)

For $a \in C^\infty(U,\mathbb R^m)$ as above, the set 
\[ 
    \mathscr J = \{(t, x) \in \mathbb R \times U | t \in J_x \}.
\]
is an open neighborhood of $\{0\} \times U$ in $\mathbb R \times U$, and the map 
\[
    \Phi : \mathscr  J \rightarrow U, \  (t, x) \mapsto \Phi(t, x)
\]
is smooth.

\end{theorem}

For a general vector field $X \in \mathfrak X(M)$ on manifolds, Equation $\dot{\gamma}(t) = X_{\gamma(t)}$ becomes $\frac{dx^i}{dt} = a^i (x(t)), i = 1,\dots,m$ after introduction of local coordinates. The existence and uniqueness theorem for ODE’s extends to manifolds, as follows:

\begin{theorem}

Let $X \in \mathfrak X(M)$ be a vector field on a manifold $M$. For any given $p \in M$, there is an open interval $\mathscr J_p \subseteq \mathbb R$ around $0$, and a solution $\gamma : \mathscr J_p \rightarrow M$ of the initial value problem 
\[
    \dot{\gamma}(t) = X_{\gamma(t)} ,\  \gamma(0) = p,
\]
which is maximal in the sense that any other solution of the initial value problem is obtained by restriction to a subinterval. The set 
\[
    \mathscr J = \{(t, p) \in\mathbb R \times M | t \in \mathscr J_p\}
\]
is an open neighborhood of $\{0\} \times M$, and the map 
\[
    \Phi : \mathscr J \rightarrow M,  \ (t, p) \mapsto  \Phi(t, p)
\]
such that $\gamma(t) = \Phi(t, p)$ solves the initial value problem mentioned above and is smooth.
\end{theorem}

Note that the uniqueness part uses the Hausdorff property in the definition of manifolds. Indeed, the uniqueness part may fail for non-Hausdorff manifolds.

\begin{definition} (Flow)

Given a vector field $X$, the map $\Phi : J \rightarrow M$ is called the flow of $X$. For any given $p$, the curve $\gamma(t) = \Phi(t, p)$ is a solution curve.  One can also fix $t$ and consider the time-$t$ flow $\Phi_t(p) \equiv \Phi(t, p).$
\end{definition}


Intuitively, $\Phi_t(p)$ is obtained from the initial point $p \in M$ by flowing for time $t$ along the vector field $X$. One expects that first flowing for time $t$, and then flowing for time $s$, should be the same as flowing for time $t +s$. Indeed one has the following
flow property

\begin{theorem}  (Flow property).

Let $X \in \mathscr X(M)$, with flow $\Phi : \mathscr J \rightarrow M$. Let $(t_2, p) \in \mathscr J$, and $t_1 \in \mathbb R$. Then 
\[ 
    (t_1,\Phi_{t_2} (p)) \in \mathscr J \iff (t_1 +t_2, p) \in \mathscr J,
\]
and one has 
\[
    \phi_{t_1} (\Phi_{t_2} (p)) = \Phi_{t_1+t_2} (p).
\]

\end{theorem}

We see in particular that for any $t$, the map $\Phi_t : U_t \rightarrow M$ is a diffeomorphism onto its image $\Phi_t(U_t) = U_{-t}$, with inverse $\Phi_{-t}$. Let $X$ be a vector field, and $\mathscr J = \mathscr J^X$ be the domain of definition for the flow $\Phi = \Phi^X$ .


\begin{definition}
A vector field $X \in \mathscr X (M)$ is called complete if $\mathscr J^X = \mathbb R \times M$. Thus $X$ is complete if and only if all solution curves exist for all time.
\end{definition}

A vector field may fail to be complete if a solution curve escapes to infinity in finite time. This suggests that a vector fields $X$ that vanishes outside a compact set must be complete, because the solution curves are 'trapped' and cannot escape to infinity:

\begin{proposition}

If $X \in \mathfrak X(M)$ is a vector field that has compact support, in the sense that $X |_{M-A} = 0$ for some compact subset $A$, then $X$ is complete. In particular, every vector field on a compact manifold is complete.
\end{proposition}

\begin{theorem}

If $X$ is a complete vector field, the flow $\Phi_t$ defines a $1$-parameter group of diffeomorphisms. That is, each $\Phi_t$ is a diffeomorphism and 
\[
    \Phi_0 = id_M, \  \Phi_{t_1} \circ \Phi_{t_2} = \Phi_{t_1+t_2}.
\]

Conversely, if $\Phi_t$ is a $1$-parameter group of diffeomorphisms such that the map $(t, p) \mapsto \Phi_t(p)$ is smooth, the equation
\[
    X_p(f) = \frac{d}{dt} \Bigr  |_{t=0} f(\Phi_{t}(p)) 
\]
defines a complete vector field $X$ on $M$, with flow $\Phi_t$.
\end{theorem}

\begin{proposition}
Let $F \in C^\infty(M,N)$, and let $X \in \mathscr X(M), Y \in \mathscr X(N)$ be complete vector fields, with flows $\Phi^X_t , \Phi^Y_t$.
\[
    X \sim_F Y \iff F \circ \Phi^X_t = \Phi^Y_t \circ F
\]
for all $t$.
\end{proposition}

In short, vector fields are $F$-related if and only if their flows are $F$-related. $\Phi^*_t : C^\infty (M) \rightarrow C^\infty (M), \Phi^* t : \mathfrak X(M) \rightarrow \mathfrak X(M)$. 


\subsection{Geometric interpretation of the Lie bracket}

For any smooth map $F \in C^\infty(M,N)$ we defined the pull-back
\[
    F^* : C ^\infty (N) \rightarrow C ^\infty (M), \ g \mapsto g \circ F.
\]
If $F$ is a diffeomorphism, then we can also pull back vector fields: $F^* : X(N) \rightarrow X(M), \ Y \mapsto F^*Y$, by the condition $(F^*Y)(F^* g) = F^* (Y(g))$ for all functions $g$. That is, $F^* Y \sim_F Y$, or in more detail $(F^*Y)_p = (T_pF)^{ -1}Y_{F(p)}$. By Theorem 5.2, we have $F^*[X,Y] = [F^*X,F^*Y]$.


Any complete vector field $X \in \mathfrak X(M)$ with flow $\Phi_t$ gives rise to a families of pull-back map.
\[
    \Phi^*_t : C ^\infty (M) \rightarrow C^\infty (M), \  \Phi^*_t : X(M) \rightarrow X(M)
\]

\begin{definition}
The Lie derivative of a function f with respect to $X$ is the function
\[
    L_X (f) = \frac{d}{dt} \Bigr |_{t=0} \Phi^*_t f ;
\]

thus $L_X (f) = X(f)$. The Lie derivative measures how $f$ changes in the direction of $X$. Similarly, for a vector field $Y$ one defines the Lie derivative  $L_X (Y)$ by
\[
   L_X (Y) = \frac{d} {dt}\Bigr |_{t=0} \Phi^*_t Y \in X(M).
\]
\end{definition}

\begin{definition}
For any $X,Y \in \mathfrak X(M)$, the Lie derivative $L_X Y$ is just the Lie bracket: $L_X (Y) = [X,Y]$.
\end{definition}

Thus, the Lie bracket $[X,Y]$ measures 'infinitesimally' how the vector field $Y$ changes along the flow of $X$. Note that in particular, $L_XY$ is skew-symmetric in $X$ and $Y$ – this is not obvious from the definition. One can also interpret the Lie bracket as measuring how the flows of $X$ and $Y$ fail to commute.

\begin{theorem}
Let $X,Y$ be complete vector fields, with flows $\Phi_t ,\Psi_s$. Then,
\begin{align*}
    [X,Y] = 0 &\iff \Phi^*_t Y = Y \text{for all t}\\
    &\iff \Psi^*_s X = X \text{for all s} \\
    &\iff  \Phi_t \circ \Psi_s  = \Psi_s \circ \Phi_t \text{for all s,t.}
\end{align*}

\end{theorem}


\subsection{Frobenius theorem}
We saw that for any vector field $X \in \mathfrak X(M)$, there are solution curves through any given point $p \in M$. The image of this curve is an (immersed) submanifold to which $X$ is everywhere tangent. One might similarly 'integral surfaces' for pairs of vector fields, and 'integral submanifolds' for collections of vector fields

\begin{definition} (Involutive)

Consider a sub-bundle $E \subseteq TM$ of rank $r$. Such a subbundle is called \textit{involutive} if the Lie bracket of any two sections of E is again a section of $E$. For vector fields $X_i$ as above, the pointwise spans 
\[
    E_p = \text{span} \{X_1|_p,\dots,X_r |_p\}
\]
define a subbundle with this property. Recall, an involution is a function that is its own inverse.
\end{definition}


\begin{definition} (Integral Submanifold)

Suppose $X_1,\dots,X_r$ are vector fields on the manifold $M$, such that the tangent vectors $X_1|_p,\dots,X_r |_p \in T_pM$ are linearly independent for all $p \in M$. A $r$-dimensional submanifold $S \subseteq M$ is called an \textit{integral submanifold} if the vector fields $X_1,\dots,X_r$ are all tangent to $S$.

Suppose that there exists an integral submanifold $S$ through any given point $p \in M$. Then each Lie bracket $[X_i ,X_j ] |_p \in T_pS$, and hence is a linear combination of $X_1|_p,\dots,X_r |_p.$ It follows that 
\[ 
    [X_i ,X_j ] = \sum_{k=1}^r c^k_{ij}X_k
\]
for certain (smooth) functions $c^k_{i j}$.

Indeed, given $X = \sum_{i=1}^m a^i X_i$ and $Y = \sum^m_{i=1} b^i X_i$ with functions $a^i ,b^i$, the condition above guarantees that $E$ is involutive. Given any rank $r$ subbundle $E \subseteq TM$ (not necessarily involutive), a submanifold $S \subseteq M$ is
called an integral submanifold if $E_p = T_pS$ for all $p \in S$.
\end{definition}


\begin{theorem} (Frobenius theorem)

Let $E \subseteq TM$ be a subbundle of rank $r$. The following are equivalent: 
\begin{enumerate}
    \item There exists an integral submanifold through every $p \in M$.
    \item $E$ is involutive. 
\end{enumerate}

In fact, if $E$ is involutive, then it is possible to find a coordinate chart $(U,\phi)$ near any given $p$, in such a way that the subbundle $(T\phi)(E|_U ) \subseteq T\phi(U)$ is spanned by the first $r \leq m$ coordinate vector fields $\frac{\partial}{\partial u^1},\dots, \frac{\partial}{\partial u^r}$.

\end{theorem}


Thus, for any involutive subbundle $E \subseteq TM$, then any $p \in M$ has an open neighborhood $U$ with a nice decomposition into $r$-dimensional submanifolds. One calls such a decomposition (or sometimes the involutive subbundle E itself) a (local) \textit{foliation}. A foliation gives a decomposition into submanifolds on a neighborhood of any given point. Globally, the integral submanifolds are often only immersed submanifolds, given by immersions $i : S \rightarrow M$ with $(T_pi)(T_pS) = E_p$ for
all $p \in S$.

\begin{example}
Let $\Phi : M \rightarrow N$ be a submersion. Then the subbundle $E \subseteq TM$ with fibers $E_p = \text{ker}(T_p \Phi) \subseteq T_pM$ is an involutive subbundle of rank $\dim M - \dim N$. Every fiber $\Phi^{-1}(q)$ is an integral submanifold.
\end{example}


\newpage
\section{Differential Forms}



\subsection{Review: Differential forms on \texorpdfstring{$\mathbb R^m$}{Rm}}

Differential forms are an approach to solving multivariable calculus problems that is independent of coordinates. They provide a unified approach to define integrands over curves, surfaces, solids, and higher-dimensional manifolds.

\begin{definition} (Wedge Product in $\mathbb R^m$)

The \textit{exterior product} or \textit{wedge product} is the product operator in an exterior algebra. If $\alpha$ and $\beta$ are differential $k$-forms of degrees $p$ and $q$, respectively, then
\[
    \alpha \wedge \beta=(-1)^{pq} \beta \wedge \alpha. 	
\]
It is not (in general) commutative, but it is associative, and bilinear. 

\end{definition}
\begin{example}
Let $\alpha, \beta \in \Omega^1(M)$. Then we define a wedge product $\alpha \wedge \beta \in \Omega^2 (M)$, as follows:
\[
    (\alpha \wedge \beta)(X,Y) = \alpha (X)\beta(Y)-\alpha(Y)\beta(X).
\]
\end{example}

\begin{definition} (Differential $k$-form)

A differential $k$-form on an open subset $U \subseteq \mathbb R^m$ is an expression of the form 
\[
    \omega = \sum_{i_1 \dots i_k} \omega i_1\dots i_k dx^{i_1} \wedge \dots \wedge  dx^{i_k}
\] 
where $\omega_{i_1\dots i_k} \in C^\infty(U)$ are functions, and the indices are numbers $1 \leq i_1 < \dots < i_k \leq m$. The symbol $\wedge$ denotes the exterior product of two differential forms.
\end{definition}

Let $\Omega^k (U)$ be the vector space consisting of such expressions, with pointwise addition. It is convenient to introduce a short hand notation $I = {i_1,\dots,i_k}$ for the index set, and write $\omega = \sum_I \omega_I dx^I$ with $\omega_I = \omega_{i_1 \dots i_k}$, and $dx^I = dx^{i1} \wedge \dots \wedge dx^{ik}$.

Since a $k$-form is determined by these functions $\omega_I$, and since there are $\frac{m!}{k!(m-k)!}$ ways of picking $k$-element subsets from $\{1,\dots,m\}$, the space $\Omega^k (U)$ can be identified with vector-valued smooth functions, $\Omega^k (U) = C^\infty (U, \mathbb R ^{\frac{m!}{k!(m-k)!}})$.

An associative product operation $\Omega^k (U) \times \Omega^l (U) \rightarrow \Omega^{k+l} (U)$ by the 'rule of computation' $dx^i \wedge d x^j = -dx^j \wedge d x^i$ for all $i, j$;  in particular $dx^i \wedge dx^i = 0$.

\begin{definition}(Exterior Differential)

Using the product structure we may define the \textit{exterior differential}
\[
    d : \Omega^k (U) \rightarrow \Omega^{k+1}(U),  \ d \bigg ( \sum_I \omega_I dx^I \bigg ) = \sum_{i=1}^m \sum_I \frac{\partial \omega_I}{\partial x^i} dx^i \wedge dx^I.
\]
\end{definition}

The key property of the exterior differential is the following fact:
\begin{proposition}
The exterior differential satisfies
\[
    d \circ d = 0,
\]
i.e. $dd\omega = 0$ for all $\omega$.
\end{proposition}


\begin{example}
Consider forms on $\mathbb R^3$
\begin{itemize}
    \item The differential of a function $f \in \Omega^0 (\mathbb R^3 )$ is a 1-form 
    \[
        df = \frac{\partial  f}{\partial x} dx + \frac{\partial f}{\partial y}  dy+ \frac{\partial f}{\partial z} dz,
    \]
    with components being the gradient, $\text{grad} f = \nabla f$.
    \item A 1-form $\omega \in \Omega^1 (\mathbb R^3 )$ is an expression $\omega = f dx+gdy+hdz$ with functions $f,g,h$. The differential is 
    \[
        d\omega = \bigg( \frac{\partial g}{\partial x} - \frac{\partial f}{\partial y} \bigg )
        dx \wedge dy +
        \bigg ( \frac{\partial h}{\partial y} - \frac{\partial g}{\partial  z} \bigg)  dy\wedge dz + \bigg(\frac{\partial  f}{\partial  z} - \frac{\partial h}{\partial x} \bigg )  dz\wedge dx.
    \] 
    Thinking of the coefficients of $\omega$ as the components of a function $F = (f,g,h) : U \rightarrow \mathbb R^3 $, we see that the coefficients of $d \omega$ give the curl of $F$, $curl(F) = \nabla \times F$.
    
    \item Finally, any 2-form $\omega \in \Omega^2 (\mathbb R^3 )$ may be written $\omega = a dy \wedge dz + b dz \wedge dx + c dx \wedge dy$, with $A = (a,b, c) : U \rightarrow \mathbb R^3$. We obtain 
    \[
        d\omega = (\frac{\partial a}{\partial x} + \frac{\partial b}{\partial y} + \frac{\partial c}{\partial z}) dx \wedge dy \wedge dz;
    \]
    the coefficient is the divergence $div(A) = \nabla A$ The usual properties $curl(grad(f)) = 0, div(curl(F)) = 0$ are both special cases of $d \circ d = 0$.
    
\end{itemize}
\end{example}

\begin{definition} (Support)

The support $supp(\omega) \subseteq U$ of a differential form is the smallest closed subset $Z$ so that $\omega$ restricted to any point in the interior of $Z$ is not identically 0. Suppose $\omega \in \Omega^m(U)$ is a compactly supported form of the top degree $k = m$, i.e. it is the set
\[
    supp(\omega )= \{p\in U : \omega_p \neq 0 \}.
\]
Such a differential form is an expression $\omega = f dx^1 \wedge \dots \wedge dx^m$ where $f \in C ^\infty(U)$ is a compactly supported function
\end{definition}

\begin{definition} (Riemann Integral)

One defines the integral of $\omega$ to be the Riemann integral:
\[ 
    \int_U \omega = \int_{\mathbb R^m} f(x^1 ,\dots, x^m )dx^1 \dots dx^m.
\]
\end{definition}

Note that we can regard $\omega$ as a form on all of $\mathbb R^m$, due to the compact support condition.

Our aim is now to define differential forms on manifolds, beginning with 1-forms. Even though 1-forms on $U \subseteq \mathbb R^m$ are identified with functions $U \rightarrow \mathbb R^m$, they should not be regarded as vector fields, since their transformation properties under coordinate changes are different. In fact, while vector fields are sections of the tangent bundle, the 1-forms are sections of its dual space, the cotangent bundle. We will thus begin with a review of dual spaces in general.

\subsection{Dual spaces}

\begin{definition} (Dual space)

For any real vector space $E$, we denote by $E^* = L(E,\mathbb R)$ (the linear subspace) as its \textit{dual space}, consisting of all linear maps $\alpha : E \rightarrow \mathbb R$.
\end{definition}

If $E$ is finite-dimensional, then the dual space is also finite-dimensional, and $\dim E^* = \dim E$. It is common to write the value of $\alpha \in E^*$ on $v \in E$ as a pairing, using the bracket notation  $\langle \alpha, v \rangle := \alpha(v);$. (In physics, it is common to use Dirac bra-ket notation $\langle \alpha | v \rangle := \alpha(v)$.)

\begin{definition} (Dual Basis)

Let $e_1,\dots, e_r$ be a basis of $E$. Any element of $E^*$ is determined by its values on these basis vectors. For $i = 1,\dots,r$, let $e^i \in E^*$ be the linear functional such that 
\[
    \langle e^i , e_j \rangle = \partial^i_j = 
    \begin{cases}
        0, & \text {if } i \neq j \\ 
        1, & \text{if } i = j
    \end{cases}
\]
The elements $e^1, \dots, e^r$ are a basis of $E^*$; this is called the \textit{dual basis}.
\end{definition}

The element $\alpha \in E^*$ is described in terms of the dual bases as $\alpha = \sum_{j=1}^r \alpha_j e^j, \  \alpha_j = \langle \alpha, e_j \rangle$. Similarly, for vectors $v \in E$ we have $v = \sum_{i=1}^r v^i e_i, \ v^i = \langle e^i , v \rangle$.

\begin{definition} (Dual Map)

Given a linear map $R : E \rightarrow F$ between vector spaces, one defines the dual map $R^* : F^* \rightarrow E^*$ (note the direction), by setting $\langle R^* \beta, v \rangle = \langle \beta ,R(v)\rangle$ for $\beta \in F^*$ and $v \in E$. 
\end{definition}

This satisfies $(R^*)^* = R$, and under the composition of linear maps, $(R_1 \circ R_2)^* = R^*_2 \circ R^*_1$. In terms of basis $e_1,\dots, e_r$ of $E$ and $f_1,\dots, f_s$ of $F$, and the corresponding dual bases (with upper indices), a linear map $R : E \rightarrow F$ is given by the matrix with entries $R_i^j = \langle f^j , R(e_i)\rangle$, while $R^*$ is described by the transpose of this matrix (the roles of $i$ and $j$ are reversed). Thus, $(R^*)^j_i = R_i^j$.

\subsection{Cotangent spaces}

\begin{definition} (Cotangent spaces, vectors, maps)

The dual of the tangent space $T_pM$ of a manifold $M$ is called the \textit{cotangent space} at $p$, denoted $T^*_p M = (T_pM)^*$.

Elements of $T^*_p M$ are called \textit{cotangent vectors}, or simply covectors. 

Given a smooth map $F \in C^\infty(M, N)$, and any $p \in M$ we have the \textit{cotangent map} $T^*_p F = (T_pF)^* : T^*_{F(p)}N \rightarrow T^*_p M$ defined as the dual to the tangent map.

\end{definition}

Thus, a co(tangent) vector at $p$ is a linear functional on the tangent space, assigning to each tangent vector at $p$ a number. The very definition of the tangent space suggests one such functional: Every function $f \in C^\infty(M)$ defines a linear map, $T_pM \rightarrow \mathbb R, v \mapsto v(f)$. This linear functional is denoted $(d f)_p \in T^*_p M$.

\begin{definition} (Differential)

Let $f \in C^\infty(M)$ and $p \in M$. The covector 
\[
    (d f)p \in T^*_p M,  \langle (d f)_p, v\rangle = v(f).
\]
is called the differential of $f$ at $p$.

\end{definition}

\begin{lemma}
For $F \in C^\infty(M,N)$ and $g \in C^\infty(N),$
\[
    d(F^* g)_p = T^*_p F((dg)_{F(p)}).
\]

Let $U \subseteq \mathbb R^m$ and $V \subseteq \mathbb R^n$ be open, with coordinates $x^1 , \dots , x^m$ and $y^1,\dots, y^n$. For $F \in C^\infty(U,V)$, the tangent map is described by the Jacobian matrix.

Thought of as matrices, the coefficients of the cotangent map are the transpose of the coefficients of the tangent map.
\end{lemma}

\subsection{1-forms}

Similar to the definition of vector fields, one can define co-vector fields, more commonly known as 1-forms: Collections of covectors $\alpha_p \in T^*_p M$ depending smoothly on the base point. 

\begin{definition} (1-form)

A 1-form on $M$ is a linear map 
\[
    \alpha : \mathfrak X(M) \rightarrow C^\infty (M), \  X \mapsto \alpha(X) = \langle \alpha, X \rangle,
\]
which is $C^\infty(M)$-linear in the sense that $\alpha(f X) = f\alpha(X)$ for all $f \in C^\infty(M), X \in \mathfrak X(M)$. The space of 1-forms is denoted $\Omega^1 (M)$.

\end{definition}

Let us verify that a 1-form can be regarded as a collection of covectors:

\begin{lemma}
Let $\alpha \in \Omega^1 (M)$ be a 1-form, and $p \in M$. Then there is a unique covector in the cotangent space $\alpha_p \in T^*_p M$ such that $\alpha(X)_p = \alpha_p(X_p)$ for all $X \in \mathfrak X(M)$. Note, we indicate the value of the function $\alpha(X)$ at $p$ by a subscript, just like we did for
vector fields.
\end{lemma}

The first example of a 1-form is described in the following definition.

\begin{definition} (Exterior differential)

The exterior differential of a function $f \in C^\infty(M)$ is the 1-form $d f \in \Omega^1 (M)$, defined in terms of its pairings with vector fields $X \in \mathfrak X(M)$ as $\langle d f, X\rangle = X(f)$.
\end{definition}


\begin{lemma}
Let $\alpha : p \mapsto \alpha p \in T^*_p M$ be a collection of covectors. Then $\alpha$ defines a 1-form, with 
\[
    \alpha(X)_p = \alpha_p(X_p)
\]
for $p \in M$, if and only if for all charts $(U,\phi)$, the coefficient functions for $\alpha$ in the chart are smooth
\end{lemma}

\subsection{Pull-backs of function and 1-forms}

Recall that for any manifold $M$, the vector space $C^\infty(M)$ of smooth functions is an algebra, with product the pointwise multiplication. Any smooth map $F : M \rightarrow M'$ between manifolds defined an algebra homomorphism, called the pull-back
\[
    F^* : C^\infty (M') \rightarrow C^\infty (M), \ \ f \mapsto F ^* (f) := f \circ F. 
\]
The fact that this preserves products is the following simple calculation: 
\[
(F^* (f)F^* (g))(p) = f(F(p))g(F(p)) = (f g)(F(p)) = F^* (f g)(p).
\]
Given another smooth map $F' : M' \rightarrow M''$ we have $(F'\circ F)^* \circ F^* \circ (F' )^*$.

Let $F \in C^\infty(M,N)$ be a smooth map. Recall that for vector fields, there is no general 'push-forward' or 'pull-back' operation, unless $F$ is a diffeomorphism. For 1-forms the situation is better: for any $p \in M$ one has the dual to the tangent map 
\[
    T^*_p F = (T_pF)^* : T^*_{F(p)}N \rightarrow T^*_p M.
\]
For a 1-form $\beta \in \Omega^1 (N)$, we can therefore define $(F^* \beta)_p := (T^*_p F)(\beta_{F(p)})$

\begin{lemma}
The collection of co-vectors $(F^*\beta)_p \in T^*_p M$ depends smoothly on $p$, defining a 1-form $F^*\beta \in \Omega^1 (M)$.
\end{lemma}

The Lemma shows that we have a well-defined pull-back map $F^* : \Omega^1 (N) \rightarrow \Omega^1  (M), \beta \mapsto F^* \beta$. Under composition of two maps, $(F_1 \circ F_2)^* = F^*_2 \circ F^*_1 $. The pull-back of forms is related to the pull-back of functions, $g \mapsto F^*g = g \circ F$:

\begin{proposition}
For $g \in C^\infty(N)$, $F^* (dg) = d(F^* g)$.
\end{proposition}

Recall once again that while $F \in C^\infty(M,N)$ induces a tangent map there is no natural push-forward operation for vector fields. By contrast, for cotangent bundles there is no naturally induced map from $T^*N$ to $T^*M$ (or the other way), yet there is a natural pull-back operation for 1-forms.  For any related vector fields $X \sim_F Y$, and $\beta \in \Omega^1 (N)$, we then have that $(F^* \beta)(X) = F^* (\beta(Y))$. Indeed, at any given $p \in M$ this just becomes the definition of the pullback map.

\subsection{Integration of 1-forms}
Given a curve $\gamma : J \rightarrow M$ in a manifold, and any 1-form $\alpha \in \Omega^1 (M)$, we can consider the pull-back $\gamma^*\alpha \in \Omega^1 (J)$. By the description of 1-forms on $\mathbb R$, this is of the form $\gamma^*\alpha = f(t)dt$ for a smooth function $f \in C^\infty(J)$.

To discuss integration, it is convenient to work with closed intervals rather than open intervals. Let $[a,b] \subseteq R$ be a closed interval. A map $\gamma : [a,b] \rightarrow M$ into a manifold will be called smooth if it extends to a smooth map from an open interval containing $[a,b]$. We will call such a map a smooth path.

\begin{definition} (Integral)

Given a smooth path $\gamma : [a,b] \rightarrow M$, we define the integral of a 1-form $\alpha \in \Gamma^1 (M)$ along $\gamma$ as
\[
\int_\gamma \alpha = \int_b^a \gamma^*\alpha.
\]
\end{definition}


The fundamental theorem of calculus has the following consequence for manifolds. It is a special case of Stokes’ theorem

\begin{proposition}
Let $\gamma : [a,b] \rightarrow M$ be a smooth path, with $\gamma(a) = p, \ \gamma(b) = q$. For any $f \in C^\infty(M)$, we have $\int_\gamma d f = f(q) -  f(p)$. In particular, the integral of $d f$ depends only on the end points of the path, rather than the path itself.

\end{proposition}

\begin{definition}

A 1-form $\alpha \in \Omega^1(M)$ such that $\alpha = d f$ for some function $f \in C^\infty(M)$ is called exact.
\end{definition}

\begin{example}
Consider the $1$-form $\alpha = y^2 e^x dx+2y e^x dy \in \Omega(\mathbb R^2)$. Find the integral of $\alpha$ along the path $\gamma  : [0,1] \rightarrow M, \ t \mapsto (sin(\pi t/2),t^3 )$. Observe that the 1-form $\alpha$ is exact: $\alpha = d (y^2 e^x) = d f$ with $f(x, y) = y^2 e^x $. The path has end points $\gamma (0) = (0,0)$ and $\gamma (1) = (1,1)$. Hence, $\int_\gamma  \alpha = f(\gamma (1))- f(\gamma (0)) = e$.
\end{example}

\subsection{2-forms}
\begin{definition}
A 2-form on $M$ is a $C^\infty(M)$-bilinear skew-symmetric map 
\[
    \alpha : \mathfrak X(M)\times \mathfrak X(M) \rightarrow C^\infty (M), (X,Y) \mapsto \alpha (X,Y)
\]

Here skew-symmetry means that $\alpha(X,Y) = -\alpha(Y,X)$ for all vector fields $X,Y$, while $C^\infty(M)$-bilinearity means 
\[
\alpha(f X,Y) = f\alpha(X,Y) = \alpha(X, fY)
\] 
for $f \in C^\infty(M)$, as well as $\alpha(X' + X'' ,Y) = \alpha(X',Y) + \alpha(X'',Y)$, and similarly in the second argument. Also, if $\alpha$ is a 2-form then so is $f\alpha$ for any smooth function $f$.
\end{definition}

\begin{example}
For an open subset $U \subseteq \mathbb R^m$, a 2-form $\omega \in \Omega^2 (U)$ is uniquely determined by its values on coordinate vector fields. By skew-symmetry the functions $\omega_{i j} = \omega  \bigg ( \frac{\partial}{\partial  x^i} , \frac{\partial}{ \partial  x^j} \bigg )$ satisfy $\omega_{i j} = -\omega_{ji}$; hence it suffices to know these functions for $i < j$. As a consequence, we see that the most general 2-form on $U$ is 
\[
    \omega  = \frac{1}{2} \sum^m_{i, j=1} \omega_{i j} dx^i \wedge dx ^j = \sum_{i<j} \omega_{i j}dx^i \wedge dx^j.
\]

\end{example}

\subsection{k-forms}
\subsubsection{Definition}
\begin{definition}
Let $k$ be a non-negative integer. A k-form on $M$ is a $C^\infty(M)$-multilinear, skew-symmetric map
\[
    \alpha : \underbrace{\mathfrak X(M) \times \dots \times \mathfrak X(M)}_\text{k times} \rightarrow C^\infty (M).
\]
The space of k-forms is denoted $\Omega^k (M)$; in particular $\Omega^0 (M) = C^\infty(M)$
\end{definition}

Here, skew-symmetry means that $\alpha(X_1,\dots, X_k)$ changes sign under exchange of any two of its elements.  The $C^\infty(M)$-multilinearity means $C^\infty(M)$-linearity in each argument, similar to the condition for 2-forms. It implies $\alpha$ is local in the sense that the value of $\alpha(X_1,\dots,X_k)$ at any given $p \in M$ depends only on the values $X_1|p,\dots,X_k |p \in T_pM$. 

If $\alpha_1,\dots,\alpha_k$ are 1-forms, then one obtains a k-form $\alpha =: \alpha_1\wedge\dots\wedge\alpha_k$ by wedge product.

Using $C^\infty$-multilinearity, a k-form on $U \subseteq R$ m is uniquely determined by its values on coordinate vector fields. i.e. by the functions,
\[
    \alpha_{i_1\dots i_k} = \alpha \bigg ( \frac{\partial}{\partial x^{i_1}}, \dots , \frac{\partial}{ \partial x ^{i_k}} \bigg )
\]
Moreover, by skew-symmetry we only need to consider ordered index sets $I = {i_1,\dots,i_k} \subseteq {1,\dots,m}$, that is, $i_1 < \dots < i_k$. Using the wedge product notation, we obtain
\[
\alpha = \sum_{i_1<\dots<i_k} \alpha_{i_1\dotsi_k} dx^{i_1} \wedge \dots dx^{i_k}.
\]

\subsubsection{Wedge product}

\begin{definition} (k,l-shuffle )

A permutation $s \in \mathfrak S_{k+l}$ is called a $k,l$-shuffle if it satisfies
\[
   s(1) < \dots < s(k), \ \ s(k +1) < \dots < s(k +l).
\]
\end{definition}

\begin{definition} (Wedge product)

The wedge product of $\alpha  \in  \Gamma^k (M)$, $\beta  \in  \Gamma^l (M)$ is the element
\[
    \alpha \wedge \beta \in  \Gamma^{k+l} (M)
\] 
given as
\[
    (\alpha \wedge\beta )(X^1,\dots ,X^{k+l}) = \sum sign(s) \alpha (X_{s(1)},\dots ,X_{s(k)}) \beta (X_{s(k+1)},...,X_{s(k+l)})
\]
where the sum is over all $k,l$-shuffles.
\end{definition}

The wedge product is graded commutative: If $\alpha \in \Omega^k (M)$ and $\beta \in \Omega^l (M)$ then $\alpha \wedge \beta = (-1) ^{kl}\beta \wedge \alpha $. Furthermore, it is associative:

\begin{proposition}
Given $\alpha_i \in \Omega_{k_i} (M)$ we have $(\alpha_1 \wedge \alpha_2)\wedge \alpha_3 = \alpha_1 \wedge (\alpha_2 \wedge \alpha_3)$
\end{proposition}

\subsubsection{Exterior differential}
Recall that we defined the exterior differential on functions by the formula $(d f)(X) = X(f)$. We will now extend this definition to all forms.

\begin{theorem}

There is a unique collection of linear maps $d : \Omega^k (M) \rightarrow \Omega^{k+1} (M)$, extending the map $(d f)(X) = X(f)$ for $k = 0$, such that $d(d f) = 0$ and satisfying the graded product rule, 
\[
    d(\alpha \wedge \beta) = d\alpha \wedge \beta + (-1) k\alpha \wedge d\beta 
\]
for $\alpha \in \Omega^k (M)$ and $\beta \in \Omega^l (M)$. This exterior differential satisfies $d \circ d = 0$.
\end{theorem}

\begin{definition} (Exact, closed k-forms)

A k-form $\omega \in \Omega^k (M)$ is called \textit{exact} if $\omega = d\alpha$ for some $\alpha \in \Omega^{k-1} (M)$. It is called closed if $d\omega = 0$.
\end{definition}

Since $d\circ d = 0$, the exact $k$-forms are a subspace of the space of closed $k$-forms; a necessary condition for $\alpha$ to be exact is that it is closed.

\begin{example}

The quotient space (closed k-forms modulo exact k-forms) is a vector space called the k-th (de Rham) cohomology
\[
H^k (M) = \frac{\{\alpha \in \Omega^k (M)| \alpha is closed \}}{\{\alpha \in \Omega^k(M)| \alpha is exact \}}.
\]

It turns out that whenever $M$ is compact (and often also if $M$ is non-compact), $H ^k (M)$ is a finite-dimensional vector space. The dimension of this vector space $b_k(M) = \dim H^k (M)$ is called the k-th Betti number of $M$; these numbers are important invariants of $M$ which one can use to distinguish non-diffeomorphic manifolds.
\end{example}

\subsection{Lie derivatives and contractions}

\begin{definition} (Contractions)

Given a vector field $X$, and a $k$-form $\alpha  \in \omega k (M)$, we can define a $k-1$-form 
\[
    \iota_X \alpha \in \omega  k-1 (M)
\] by \textit{contraction}: Thinking of $\alpha $ as a multi-linear form, one simply puts $X$ into the first slot:
\[
    (\iota_X\alpha )(X_1,...,X_{k-1}) = \alpha (X,X_1,\dots,X_{k-1}).
\]
Contractions have the following compatibility with the wedge product, similar to that for the exterior differential: 
\[
    \iota_X (\alpha  \wedge \beta ) = \iota_X\alpha  \wedge \beta  + (-1) k\alpha  \wedge \iota_X \beta , 
\]
for $\alpha  \in \omega^k (M)$,$\beta  \in \omega^l (M)$, which one verifies by evaluating both sides on vector fields.
\end{definition}

Another important operator on forms is the Lie derivative:

\begin{theorem}
Given a vector field $X$, there is a unique collection of linear maps $L_X : \Omega^k (M) \rightarrow \Omega^k (M)$, such that 
\[ 
    L_X (f) = X(f), \ L_X (d f) = dX(f),
\] 
and satisfying the product rule, 
\[
L_X (\alpha  \wedge \beta ) = L_X \alpha  \wedge \beta  +\alpha  \wedge L_X \beta  
\] 
for $\alpha  \in \Omega^k (M)$ and $\beta  \in \Omega^l (M)$.
\end{theorem}

These operators, $d, L_X, \iota_X$, have the following compatibilities with the wedge product: For $\alpha \in \Omega^k (M)$ and $\beta \in \Omega^l (M)$ one has 
\begin{align*}
d(\alpha \wedge\beta) &= (d\alpha)\wedge\beta + (-1) k\alpha \wedge d\beta,\\
L_X (\alpha \wedge\beta) &= (L_X\alpha)\wedge\beta +\alpha \wedge L_X \beta,\\
\iota_X (\alpha \wedge\beta) &= (\iota_X\alpha)\wedge\beta + (-1) k\alpha \wedge\iota_X \beta.
\end{align*}
  
One says that $L_X$ is an \textit{even derivation} relative to the wedge product, whereas $d,\iota_X$ are \textit{odd derivations}. They also satisfy important relations among each other:
\begin{align*}
    d \circ d = 0 \\
    L_X \circ L_Y -L_Y \circ L_X = L_{[X,Y]} \\
    \iota_X \circ \iota_Y +\iota_Y \circ \iota_X = 0 \\
    d \circ L_X -L_X \circ d = 0 \\
    L_X \circ \iota_Y -\iota_Y \circ L_X = \iota_[X,Y] \\
    \iota_X \circ d+d \circ \iota_X = L_X .
\end{align*}
This collection of identities is referred to as the Cartan calculus, , and in particular the last identity is called the Cartan formula.

\subsubsection{Pull-backs}

\begin{definition} (k-form Pullbacks)

Similar to the pull-back of functions (0-forms) and 1-forms, we have a pull-back operation for k-forms, $F^* : \Omega^k (N) \rightarrow \Omega^k (M)$ for any smooth map between manifolds, $F \in C^\infty(M,N)$. Its evaluation at any $p \in M$ is given by
\[
(F^* \beta )_p(v_1,\dots, v_k) = \beta_{F(p)} (T_pF(v1),\dots,T_pF(v_k)).
\]
\end{definition}

The pull-back map satisfies $d(F^*\beta) = F^*d\beta$, and for a wedge product of forms, $F^* (\beta_1 \wedge \beta_2) = F^* \beta_1 \wedge F^* \beta_2$.

\begin{proposition}

Let $U \subseteq \mathbb R^m $with coordinates $x^i$ , and $V \subseteq \mathbb R^n $ with coordinates $y^j $. Suppose $m = j$, and $F \in C^\infty(U,V)$. Then 
\[
F^* (dy^1 \wedge \dots \wedge dy^n ) = J dx^ 1 \wedge \dots \wedge dx^n
\]
where $J(x)$ is the determinant of the Jacobian matrix, 
\[
    J(x) = \det \bigg (\frac{\partial F^i}{\partial  x^j} \bigg ) ^n_{i, j=1}.
\]

\end{proposition}

The Lie derivative $L_X \alpha $ of a differential form with respect to a vector field $X$ has an important interpretation in terms of the flow $\Phi_t$ of $X$. Assuming for simplicity that $X$ is complete (so that $\Phi_t$ is a globally defined diffeomorphism), one has the formula 
\[
    L_X\alpha = \frac{d}{dt} \Bigr |_{t=0} \Phi^*_t \alpha.
\]
The formula shows that $L_X$ measures to what extent $\alpha$ is invariant under the flow of $X$.

\subsubsection{Integration of differential forms}

Differential forms of top degree can be integrated over oriented manifolds. Let $M$ be an oriented manifold of dimension $m$, and $\omega \in \Omega^m(M)$. Let $supp(\omega)$ be the support of $\omega$. If $supp(\omega)$ is contained in an oriented coordinate chart $(U,\phi)$, then one defines
\[
\int_M \omega = \int_{\mathbb R^m} f(x)dx^1\cdots dx^m 
\]

where $f \in C^\infty(\mathbb R^m)$ is the function, with $supp(f) \subseteq \phi(U)$, determined from 
\[
(\phi^{-1} )^* \omega = f dx^1 \wedge \cdots \wedge dx^m.
\]
This definition does not depend on the choice of oriented chart.

If $\omega$ is not necessarily supported in a single oriented chart, we proceed as follows. Let  $ (U_i , \phi^i), \ i = 1,\dots,r$ be a finite collection of oriented charts covering $supp(\omega)$. Together with $U_0 = M \setminus supp(\omega)$ this is an open cover of $M$.

\begin{lemma}
Given a finite open cover of a manifold there exists a partition of unity subordinate to the cover, i.e. functions $\chi_i \in C^\infty(M)$ with $supp(\chi_i) \subseteq U_i$ and $\sum_{i=0}^r \chi_i = 1$.

\end{lemma}

Indeed, partitions of unity exists for any open cover, not only finite ones. Let $\chi_0, \dots, \chi_r$ be a partition of unity subordinate to this cover. We define
\[
    \int_M \omega = \sum_{i=1}^r \int_M \chi_i \omega
\] 
where the summands are defined as above, since $\chi_i\omega$ is supported in $U_i$ for $i \geq 1$. It can be shown that this is well defined, independent of the choice of oriented coordinate charts.

\subsubsection{Integration over oriented submanifolds}

Let $M$ be a manifold, not necessarily oriented, and $S$ is a k-dimensional oriented submanifold, with inclusion $i : S \rightarrow M$. We define the integral over $S$, of any k-form $\omega \in \Omega^k (M)$ such that $S \cap supp(\omega)$ is compact, as follows:

\[
    \int_S \omega = \int_S i^* \omega.
\]
Of course, this definition works equally well for any smooth map from $S$ into $M$. For example, the integral of compactly supported 1-forms along arbitrary paths $\gamma : \mathbb R \rightarrow M$ is defined. Note also that $M$ itself does not have to be oriented, it suffices that $S$ is oriented.

\subsubsection{Stokes’ theorem}
Let $M$ be an $m$-dimensional oriented manifold.

\begin{definition} (Boundary and interior of region)

A region with (smooth) boundary in $M$ is a closed subset $D \subseteq M$ with the following property: There exists a smooth function $f \in C^\infty(M,R)$ such that $0$ is a regular value of $f$ , and
\[
    D = \{p \in M | f(p) \leq 0 \}.
\]
We do not consider $f$ itself as part of the definition of $D$, only the existence of $f$ is required. 

The interior of a region with boundary, given as the largest open subset contained in $D$, is 
\[
   int(D) = \{p \in M| f(p) < 0,
\]
and the boundary itself is
\[
   \partial D = \{p \in M| f(p) = 0 \},
\]
a codimension 1 submanifold (i.e., hypersurface) in $M$.

\end{definition}


Recall that we are considering $D$ inside an oriented manifold $M$. The boundary $\partial D$ may be covered by oriented submanifold charts $(U,\phi)$, in such a way that $\partial D$ is given in the chart by the condition $x^1 = 0$, and $D$ by the condition $x^1 \leq 0$:
\[ 
    \phi (U \cap D) = \phi (U) \cap \{x \in \mathbb R^m | x ^1 \leq 0 \}.
\]

We call oriented submanifold charts of this kind \textit{'region charts'}.

\begin{lemma}
The restriction of the region charts to $\partial D$ form an oriented atlas for $\partial D$.
\end{lemma}
In particular, $\partial D$ is again an oriented manifold. To repeat: If $x^1 ,\dots, x^m$ are local coordinates near $p \in \partial D$, compatible with the orientation and such that $D$ lies on the side $x^1 \leq 0$, then $x^2 ,\dots, x^m$ are local coordinates on $\partial D$. This convention of ‘induced orientation’ is arranged in such a way that the Stokes’ theorem holds without extra signs.

For an $m$-form $\omega$ such that $supp(\omega)\cap D$ is compact, the integral $\int_D \omega $ is defined similar to the case of $D = M$.

\begin{theorem} (Stokes' Theorem)


Let $M$ be an oriented manifold of dimension $m$, and $D \subseteq M$ a region with smooth boundary $\partial D$. Let $\alpha \in  \Omega^{m-1} (M)$ be a form of degree $m-1$, such that $supp(\alpha)\cap D$ is compact. Then 
\[
\int_D d\alpha = \int_{\partial D} \alpha.
\]
\end{theorem}

As explained above, the right hand side means $\int_{\partial D} i^* \alpha $, where $i : \partial D \rightarrow M$ is the inclusion map.

\begin{corollary}
Let $\alpha \int \Omega^{m-1}(M)$ be a compactly supported form on the oriented manifold $M$. Then
\[ 
    \int_M d\alpha = 0.
\]
\end{corollary}

Note that it does not suffice that $d\alpha$ has compact support. A typical application of Stokes’ theorem shows that for a closed form $\omega \in \Omega^k (M)$, the integral of $\omega$ over an oriented compact submanifold does not change with smooth deformations of the submanifold.

\begin{theorem}

Let $\omega \in \Omega^k (M)$ be a closed form on a manifold $M$, and $S$ a compact, oriented manifold of dimension $k$. Let $F \in C^\infty(R\times S,M)$ be a smooth map, thought of as a smooth family of maps 
\[
F_t = F(t, \cdot) : S \rightarrow M.
\]
Then the integrals $\int_S F^* t \omega$ do not depend on $t$.
\end{theorem}

If $F_t$ is an embedding, then this is the integral of $\omega$ over the submanifold $F_t(S) \subseteq M$.

\begin{definition} (Smooth isotopy)

Given a smooth map $\phi : S \rightarrow M$, one refers to a smooth map $F : R \times S \rightarrow M$ with $F_0 = \phi $ as an \textit{smooth deformation or isotopy} of $\phi$. We say that $\phi$ can be smoothly deformed into $\phi'$ if there exists a smooth isotopy $F$ with $\phi = F_0$ and $\phi' = F_1$. 
\end{definition}

The previous theorem shows that if $S$ is oriented, and if there is a closed form $\omega \in \Omega^k (M)$ with
\[
 \int_S \phi^* \omega \neq \int_S (\phi' )^* \omega 
\]
then $\phi$ cannot be smoothly deformed into $\phi 0$.


\subsubsection{Volume forms}

\begin{definition}

A non vanishing 1-form $\alpha$ at point $p$ means that there is a vector $v$ in $T_pM$ such that $\alpha_p(v)\neq 0$. Similarly for the $k$-form, it means that there is a set of $k$ vectors such the form is nonzero if evaluated on these vectors.
\end{definition}

\begin{definition} (Volume form)

A top degree differential form $\Gamma \in \Omega^m(M)$ is called a \textit{volume form} if it is nonvanishing  everywhere: $\Gamma_p \neq 0$ for all $p \in M$. In a local coordinate chart $(U,\phi)$, this means that 
\[ (\phi^{-1} )^*\Gamma = f dx^1 \wedge \cdots \wedge dx^m 
\]
where $f(x) \neq 0$ for all $x \in  \phi(U)$.

\end{definition}


\begin{lemma}

A volume form $\Gamma \in \Omega^m(M)$ determines an orientation on $M$, by taking as the oriented charts those charts $(U,\phi)$ such that 

\[
    (\phi^{-1})^*\Gamma = f dx^1 \wedge \cdots \wedge dx^m
\]
with $f > 0$ everywhere on $\Phi(U)$.
\end{lemma}


\begin{theorem}

A manifold $M$ is orientable if and only if it admits a volume form. In this case, any two volume forms compatible with the orientation differ by an everywhere positive smooth function: 
\[
    \Gamma' = f \Gamma , f > 0.
\]

\end{theorem}

\begin{definition} (Volume)

For a compact manifold $M$ with a given volume form $\Gamma \in \Omega^m(M)$, one can define the volume of $M$, 
\[
vol(M) = \int_M \Gamma.
\]
Here the orientation used in the definition of the integral is taken to be the orientation given by $\Gamma$ . Thus $vol(M) > 0$.
\end{definition}

Note that volume forms are always closed, for degree reasons (since $\Omega^{m+1} (M) = 0$). But on a compact manifold, they cannot be exact:

\begin{theorem}

Let $M$ be a compact manifold with a volume form $\Gamma \in \Omega^m(M)$. Then $\Gamma$ cannot be exact.
\end{theorem}

\subsection{Cartan calculus}


\subsection{Cohomology Groups}

\newpage
\section{Riemannian Geometry}

\newpage
\section{Lie Groups}


% The fundamental objects of study in differential geometry are manifolds. Roughly, an n-dimensional manifold is a mathematical object that ``locally'' looks like $\mathbb R^n$.  Manifolds in euclidean space are described with a \textit{regular level set}, $S = f^{-1}(a)$ which defines a smooth hypersurface $S \subseteq R^n$. For example, the n-dimensional sphere described by:
% \[
%     S^n = \{ (x^0, \dots, x^n) \in \mathcal R^{n+1} | (x^0)^2 + \dots + (x^n)^2  = 1\}.
% \]
% Another example is the 2-Torus, $T^2$. Given real numbers $r, R$ with $0 < r < R$, take a circle of radius $r$ in the $x-z$ plane, with center at $(R,0)$, and rotate about the $z$-axis:
% \[
%     T^2 = \{ (x,y,z) | (\sqrt{x^2 + y^2} - R)^2 + z^2 + r\}
% \]

% The sphere, the torus, the double torus, triple torus, and so on are ‘orientable’ surfaces, which essentially means that they have two sides which you might paint in two different colors. It turns out that these are all orientable surfaces, if we consider the surfaces ‘intrinsically’ and only consider surfaces that are compact in the sense that they don’t go off to infinity and do not have a boundary (thus excluding a cylinder, for example).

% Not all surfaces can be realized as ‘embedded’ in $\mathbb R^3$; for \textit{non-orientable surfaces} one needs to allow for self-intersections. This type of realization is referred to as an immersion: We don’t allow edges or corners, but we do allow that different parts of the surface pass through each other.  An example is the Klein bottle, which is not possible to represent as a regular level set $f^{-1}(0)$ of a function $f$ since any suface has one side where $f$ is positive and another side where $f$ is negative.

% The projective plane or projective space is denoted $\mathbb{R} P^2$ and is defined as the set of all lines (i.e., 1-dimensional subspaces) in $\mathbb{R}^3$. we can also think of $\mathbb{R} P^2$ as the set of antipodal (i.e., opposite) points on $S^2$. Splitting the points into those with distance $< \epsilon$ from the equator and those $\geq \epsilon$ produces a Mobius strip and a two-dimensional disc. Generating a smooth curve by gluing the boundary of a Mobius strip to the boundary of a disk is depicted in what's known as Boy’s surface.

% Another operation for surfaces, generalizing the procedure of ‘attaching handles’, is the connected sum Given two surfaces $\sigma_1$ and $\sigma_2$, remove small disks around given points $p_1 \in \sigma_1$ and $p_2 \in \sigma_2$, to create two surfaces with boundary circles. Then glue-in a cylinder connecting the two boundary circles, without creating edges. The resulting surface is denoted $\sigma_1\#\sigma_2$.

% It turns out that all closed, connected surfaces are obtained from either the 2-sphere $S^2$, the Klein bottle, or  $\mathbb{R} P^2$, by attaching handles with the connected sum.

% “gauge equivariance.” This means that quantities in the world and their relationships don’t depend on arbitrary frames of reference (or “gauges”); they remain consistent whether an observer is moving or standing still, and no matter how far apart the numbers are on a ruler. Measurements made in those different gauges must be convertible into each other in a way that preserves the underlying relationships between things.

\newpage
\begin{thebibliography}{}

\bibitem[]{}
E. Meinrenken and G. Gross, Introduction to Differential Geometry, Lecture Notes for MAT367. 
%http://www.math.utoronto.ca/~mein/teaching/MAT367/DiffGeomNotes.pdf

\bibitem[]{}
Samelson, H. Review: Werner Greub, Stephen Halperin and Ray Vanstone, Connections, curvature, and cohomology. Bull. Amer. Math. Soc. 83 (1977), no. 5, 1011--1015. https://projecteuclid.org/euclid.bams/1183539466
%http://im0.p.lodz.pl/~kubarski/AnalizaIV/Wyklady/GHV/ITOM/G-H-V-1%20Connections,%20Curvature,%20and%20Cohomology.pdf

\bibitem[]{}
Paul Seidel. 18.950 Differential Geometry. Fall 2008. Massachusetts Institute of Technology: MIT OpenCourseWare, https://ocw.mit.edu. License: Creative Commons BY-NC-SA.
%https://ocw.mit.edu/courses/mathematics/18-950-differential-geometry-fall-2008/index.htm#

\bibitem[]{}
Sigurdur Helgason. 18.755 Introduction to Lie Groups. Fall 2004. Massachusetts Institute of Technology: MIT OpenCourseWare, https://ocw.mit.edu. License: Creative Commons BY-NC-SA.
%https://ocw.mit.edu/courses/mathematics/18-755-introduction-to-lie-groups-fall-2004/index.htm#


\end{thebibliography}


\end{document}
